\chapter{Analýza aplikace}

V této kapitole nejprve identifikujeme vlastní problematiku čteček, jíž se budeme dále zabývat, a popíšeme požadavky na aplikaci, která bude dané problémy řešit.

\section{Identifikace problému}

Naší snahou je pokrýt jedinou aplikací několik příbuzných problémů, kterým čelí náročný uživatel.
Náročný uživatel má podle nás především následující potřeby.

\subsection{Přístup odkudkoli}
Z vlastní zkušenosti víme, že náročný uživatel často střídá zařízení, na kterých prochází internet.
Může to být domácí počítač, notebook, mobilní telefon, tablet; uživatel chce mít přístup ke čtečce odkudkoli.
Vyžaduje, aby bylo možné přejít z jednoho zařízení na druhé a na něm pokračovat ve čtení článků.

Hledáme tedy webovou službu, která bude zajišťovat synchronizaci a dostupnost bez ohledu na zařízení, které uživatel aktuálně používá.

\subsection{Kanály RSS, Atom}
Náročný uživatel chce/musí sledovat několik desítek různých webových stránek, blogů, portálů.
Naštěstí, většina z nich poskytuje informace o novinkách (nových článcích, které byly publikovány) pomocí kanálů RSS nebo Atom.
Toto je také nejčastější využití hledané čtečky; vyhovuje mu téměř každá čtečka, kterou jsme popsali v minulé kapitole.

Čtečka musí podporovat získáváni informací z kanálů RSS a Atom.

\subsection{Stránky bez RSS a Atomu}
Mnoho webových stránek neposkytuje kanály RSS ani Atom; jedná se často o různé blogy, z akademického prostředí jsou to například stránky přednášek nebo cvičení.
Existují služby, které umožňují sledovat webovou stránku a v případě detekování její změny nás informují e-mailem; takovou službou je například \projekt{\mbox{ChangeDetection}}{http://www.changedetection.com/}.
Přestože tato služba může svoji úlohu vykonávat dobře, neumožňuje integraci s jinými systémy.

Zmíníme ještě snahu studentů, kteří si často píší jednoúčelové skripty, které periodicky stahují webovou stránku cvičení a porovnávají její obsah s minulou verzí.
Jejich přístup ale trpí stejným problémem jako výše zmíněná služba.

Často nás nezajímá samotná změna stránky, ale důvod, proč k ní došlo.
Pokud publikujeme nový článek na webu (vytvoříme tedy novou stránku), dáme o něm vědět (přidáme odkaz) většinou v seznamu nejnovějších článků na úvodní stránce.
Pokud vyjdeme z tohoto předpokladu, nezajímá nás přítomnost nového odstavce na stránce, ale odkaz na samotný nový článek, který je v odstavci zmíněn.

Nejsme si vědomi, že by existovala služba, která by umožňovala detekci nově publikovaných článků pomocí rozboru samotných odkazů na webové stránce.

\subsection{Uložení adresy článku}

Pokud si uživatel chce uložit nějaký článek, resp. jeho internetovou adresu, typicky má na výběr několik různých služeb.

Uživatel si může uložit adresu webové stránky do záložek v prohlížeči, nicméně jejich synchronizace je problematická.
Výrobci prohlížečů sice často umožňují jejich synchronizaci, ale ta je podmíněná tím, že uživatel používá všude prohlížeč od stejného výrobce.
Synchronizaci záložek umožňují také různé služby, které nabízejí doplňky do prohlížečů.
Některé z nich fungují dokonce nezávisle na konkrétním prohlížeči, nicméně je třeba je mít nainstalované, jinak pohodlí používání těchto služeb dramaticky klesá.
Příkladem je služba \projekt{Delicious}{http://delicious.com/}.

S těmito službami souvisí tlačítka, která najdeme všude na internetu obvykle pod názvy \uv{share button}.
Jejich příkladem může být plugin \projekt{AddToAny}{http://wordpress.org/plugins/add-to-any/} do redakčního systému \projekt{Wordpress}{http://wordpress.org/}.
Použitím těchto tlačítek je možné odeslat články do některé z rozšířenějších služeb; často je jejich seznam omezený na dvě nejznámější sociální sítě.
Počet webových portálů, které ukládání článků tímto způsobem umožňují, je značně omezený, určitě na tuto vlastnost nelze obecně spoléhat.

Čtečky RSS tuto problematiku také částečně řeší, umožňují k položkám přidávat hvězdičky a štítky.
Problémem ale stále zůstává fakt, že štítek je možné přidat pouze k článku, který je ve čtečce dostupný.
Obvykle není možné přidat libovolnou vlastní internetovou adresu do RSS čtečky tak, aby s ní čtečka pracovala stejným způsobem, jako s ostatními položkami.

Jelikož je naše aplikace povahou čtečka, budeme vyžadovat možnost jednoduchého přidání internetové adresy jako nové položky.

\subsection{Přečíst později}

Často si uživatelé nemohou přečíst zajímavý článek okamžitě, protože na to nemají čas, nebo se nenacházejí ve vhodném prostředí.
Důvodem může být to, že článek je příliš dlouhý nebo obsahuje video.
V takových případech si chtějí uživatelé článek uložit/poznamenat k pozdějšímu přečtení.

Jednou z možností je využít některou ze služeb zmíněných v minulém bodu.
Další možností, které uživatelé využívají, jsou specializované služby.
Tyto služby se často snaží přeformátovat článek do podoby vhodnější ke čtení (odstraňují rušivé elementy).
Zástupcem je například \projekt{Instapaper}{http://www.instapaper.com/} či \projekt{Pocket}{http://getpocket.com/}.

\subsection{Efektivní ovládání}

Aby náročný uživatel mohl čtečku efektivně používat, musí mu aplikace nabídnout adekvátní prostředí.
Nároky, které na toto prostředí uživatel klade, jsme již uvedli v části~\ref{ssec:narocny-uzivatel}.
Zjistili jsme, že efektivního ovládání aplikace může uživatel docílit používáním klávesových zkratek.

Ke klávesovým zkratkám ještě zmíníme několik poznámek.
Nejčastější operace musí být snadno vyvolatelné, ideálně stisknutím jediného tlačítka.
Uživatel by měl mít možnost nastavit si své vlastní schéma klávesových zkratek.

\bigskip

Uvedeme ještě dvě další potřeby, které nejsou tak důležité, jako výše zmíněné, nicméně jsou jejich přirozeným rozšířením.

\subsection{Zálohování článku}

Občas se stane, že webový portál, na němž byl publikovaný zajímavý a pro uživatele důležitý článek, ukončí svůj provoz.
V takové situaci je uživatel většinou rozladěn a následně hledá, zda je článek dostupný v internetových archivech.

Aby tomu předešel, mohl by si zazálohovat aktuální podobu článku pro případ, že originál nebude dostupný.
Všechny prohlížeče umožňují uložit webovou stránku na disk, nicméně správa takto uložených stránek určitě není pohodlná.

Služba, kterou hledáme, by měla umožnit zazálohovat konkrétní webovou stránku a sama jí poskytovat v zazálohované podobě pro případ, že originál nebude dostupný.

\subsection{Sdílení článků}

Uživatelé mezi sebou nejčastěji sdílejí odkazy na zajímavé články pomocí sociálních sítí; ku příkladu \projekt{Twitter}{http://twitter.com/} je na tomto principu prakticky založen.
Sociální sítě umožňují sledovat několik zajímavých osob a zobrazovat obsah, který se jim líbí či jej doporučují.

Těmto službám nechceme konkurovat, chceme nabídnout odlišný přístup ke sdílení.
Nabídnout uživatelům možnost nenásilně si vytvořit několik vlastních kanálů, třeba tematicky zaměřených, které mohou ostatní odebírat bez ohledu na aplikaci, kterou používají.

\bigskip

Přestože existuje mnoho aplikací, které naplňují jednotlivé potřeby, tak jak jsme je zde popsali, nepodařilo se nám najít žádnou takovou, která by komplexně řešila celou výše zmíněnou problematiku.

\section{Zavedení pojmů}

Před samotným rozborem problému zavedeme několik pojmů, které budeme používat v následujících částech dokumentu.
Jednotlivým termínům bude později odpovídat jedna entita.

\begin{description}
    \item[Uživatel] Osoba, která používá naši aplikaci.
		Musíme být schopni rozlišit jednotlivé uživatele a poskytnout jim různý obsah.
	\item[Zdroj] Zdroj je reprezentací jednoho dokumentu (\zkratka{HTML}{HyperText Markup Language -- hlavní značkovací jazyk používaný při tvorbě webových stránek~\cite{Raggett:1999}}, \zkratka{XML}{Extensible Markup Language -- univerzální strojově i lidsky čitelný značkovací jazyk určený pro ukládání a výměnu dat~\cite{Bray:1998}}) v síti internet, který naše aplikace používá k získávání položek.
		Zdrojem může být webová stránka nebo některý ze systémů poskytování informací o novinkách.
	\item[Kontrola zdroje] Úkon provedený k zjištění, zda zdroj obsahuje nové položky.
		Obvykle je kontrola realizovaná stáhnutím dokumentu reprezentovaného zdrojem a porovnáním s jeho minulou verzí.
    \item[Položka] Položka reprezentuje odkaz na jeden článek nebo stránku v síti internet.
		Každá položka je vázaná právě k jednomu zdroji.
    \item[Štítek] Popisek, který může uživatel přidat k položce.
		V našem případě bude popisek krátký, většinou jednoslovný.
\end{description}

\section{Analýza požadavků}

Identifikované problémy popíšeme z různých pohledů vlastními požadavky, které bude naše aplikace splňovat.

\subsection{Funkční požadavky}

Mezi požadavky na funkcionalitu, které musí aplikace splňovat, patří následující:

\begin{itemize}
    \item Každý uživatel aplikace si může zaregistrovat svoje zdroje, které bude sledovat.
    \item Existuje několik typů zdrojů: zdroje, které přebírají položky z kanálu RSS nebo Atom a zdroje, které parsují informace o novinkách přímo z webových stránek.
    \item Každý uživatel si bude moci do aplikace přidávat svoje položky.
    \item Seznam všech položek bude možné filtrovat podle několika kritérií.
	\item Uživatel může zazálohovat jakoukoli webovou stránku.
	\item Uživatel má možnost sdílet svoje články.
\end{itemize}

\subsection{Požadavky na použitelnost}

Požadavky na použitelnost popisují, jak má aplikace vypadat a jak se má chovat, aby se pohodlně používala a ovládala.

\begin{itemize}
    \item Aplikace musí být ovladatelná klávesovými zkratkami.
    \item Největší část aplikace bude zaujímat seznam položek.
    \item Každá položka obsahuje odkaz na originální webovou stránku.
    \item Aplikace bude fungovat ve všech běžných prohlížečích.
\end{itemize}

\subsection{Výkonnostní požadavky}

Výkonnostní požadavky určují parametry zátěže, na kterou má být aplikace navržena.

\begin{itemize}
	\item Aplikaci může používat několik stovek uživatelů.
    \item Každý uživatel může mít desítky až stovky zdrojů.
	\item Zdroj může poskytovat desítky položek denně.
    \item Každý zdroj se kontroluje přiměřeně často v závislosti na frekvenci změn.
    \item Aplikace bude dobře škálovatelná.
\end{itemize}

\section{Moduly aplikace}

Z požadavků, které na aplikaci klademe, vyplývají důsledky na volbu její architektury.
Aplikace se bude skládat ze tří modulů, které spolu budou vzájemně komunikovat.

\subsection{Serverová část}

Serverová část aplikace bude zodpovědná za získávání nových položek z jednotlivých zdrojů, uložení všech dat včetně uživatelského nastavení a poskytování těchto dat ostatním modulům.

\subsection{Klientská část}

Klientská část bude zobrazovat data získaná ze serverové části a bude interagovat s uživatelem, který pomocí ní aplikaci ovládá.
Půjde o standardní webové stránky fungující v běžných prohlížečích.

\subsection{Doplněk do prohlížeče}

Doplněk bude tvořit nepovinné rozšíření do nejběžnějších prohlížečů; bude mít za cíl zjednodušit práci s aplikací.
Tento doplněk nebude poskytovat žádnou vyšší funkcionalitu, které by nešlo docílit klientskou částí.
Bude umožňovat uživateli přidat novou položku do svého manuálního zdroje pouhými dvěma kliknutími myší.

