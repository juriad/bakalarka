\chapter{Analýza}

V této kapitole nejprve identifikujeme vlastní problémy čteček, jímiž se budeme dále zabývat a popíšeme požadavky na aplikaci, která bude dané problémy řešit.

\section{Identifikace problému}

Naší snahou je pokrýt jednou aplikací několik příbuzných problémů, kterým čelí náročný uživatel.

Náročný uživatel má především potřebu:
% TODO linky
\begin{enumerate}
    \item mít přístup k seznamu článků odkudkoli.
        Hledáme webovou službu, která bude zajišťovat synchronizaci a dostupnost bez ohledu na zařízení, které uživatel aktuálně používá.
    \item sledovat několik webových stránek, které poskytují kanál RSS nebo Atom.
        Tomuto bodu vyhovuje téměř každá čtečka ve formě webové služby, kterou jsme popsali v minulé kapitole.
    \item sledovat takové webové stránky, které neposkytují kanál RSS ani Atom.
        Toto splňuje jen několik málo speciálních služeb, které většinou neposkytují uživateli komfort jednotného prostředí --- informují uživatele o změnách jen e-mailem.
		Takovou službou je například http://www.changedetection.com/.
    \item uložit si informaci o zajímavých článcích.
        Tomuto požadavku by mohly vyhovovat záložky v prohlížeči, jenže ty zase nesplňují možnost přístupu odkudkoli.
        To se snažila změnit služba https://delicious.com/, ale ta zase nespňuje předchozí dva body.
        Čtečky z bodu č. 2 tuto problematiku také částečně řeší, umožňují ke svým položkám přidávat hvězdičky a štítky; nicméně, obvykle není možné přidat štítek k článku, který ve čtečce není.
    \item uložit si informaci o článcích, které si chce přečíst později.
        Tento bod je velice podobný předchozímu, jen mění sémantiku toho, které články jsou takto označeny; služby, které to umožňují jsou třeba http://getpocket.com/ nebo http://www.instapaper.com/.
    \item najít a přečíst článek, který si dříve označil.
        Tento bod řeší všechny aplikace, které řeší předchozí body.
    \item ovládat aplikaci efektivně.
        Tento bod vyloženě vyžaduje maximální efektivitu a rychlost práce s aplikací, což v našem kontexu znamená existenci klávesových zkratek.
\end{enumerate}

Uvedeme ještě dvě další potřeby, které nejsou tak důležité jako výše zmíněné, nicméně jsou jejich přirozeným rozšířením.
Mezi tyto potřeby patří:

\begin{enumerate}
	\setcounter{enumi}{7}
	\item zálohovat článek.
		Uživatel by měl mít možnost zazálohovat aktuální podobu článku pro případ, že originál nebude dostupný.
		Opět existují služby, které toto umožňují, mezi nejznámější patří http://archive.org/ a cache vyhledávače Google.
	\item sdílet své položky.
		Aplikací, které umožňují sdílet položky (odkazy na zajímavé články) s jinými uživateli existuje mnoho; sociální síť Twitter je na tomto principu prakticky založená.
\end{enumerate}

\begin{leftbar}
	Nenašel jsem žádnou existující aplikaci, která by řešila celou výše zmíněnou problematiku.
\end{leftbar}

\section{Zavedení pojmů}

Před samotným rozborem problému zavedeme několi pojmů, které budeme používat ve zbytku dokumentu.
Jednotlivým termínům bude později odpovídat jedna entita.

\begin{description}
    \item[Uživatel] Osoba, která používá naši aplikaci.
    \item[Zdroj] Zdroj je reprezentací jednoho dokumentu (HTML či XML) v síti internet, který naše aplikace používá k získávání položek.
    \item[Položka] Položka reprezentuje odkaz na jeden článek nebo stránku v síti internet.
    \item[Štítek] Nejmenší informace, kterou může uživatel přidat k položce.
\end{description}

\section{Analýza požadavků}

Identifikované problémy popíšeme z různých pohledů vlastními požadavky, které bude naše aplikace splňovat.

\subsection{Funkční požadavky}

Mezi požadavky na funkcionalitu, které musí aplikace splňovat patří následující:

\begin{itemize}
    \item Každý uživatel aplikace si může zaregistrovat své zdroje, které bude sledovat.
    \item Existuje několik typů zdrojů: Zdroje, které přebírají položky z kanálu RSS nebo Atom, zdroje, které parsují informace o novinkách přímo z webových stránek a manuální zdroje.
    \item Každý uživatel bude mít vlastní manuální zdroj, který nebude sám o sobě poskytovat nové položky, ale bude obsahovat takové položky, které si uživatel vloží manuálně.
    \item Seznam všech položek bude možné filtrovat podle několika kritérií.
	\item Uživatel může zazálohovat jakoukoli webovou stránku.
	\item Uživatel má možnost sdílet své články.
\end{itemize}

\subsection{Požadavky na použitelnost}

Požadavky na použitelnost popisují, jak má aplikace vypadat a jak se má chovat, aby se pohodlně používala a ovládala.

\begin{itemize}
    \item Aplikace musí být ovladatelená klávesovými zkratkami.
    \item Největší část aplikace bude zaujímat seznam položek.
    \item Každá položka obsahuje odkaz na originální webovou stránku.
    \item Aplikace bude fungovat ve všech běžných prohlížečích.
\end{itemize}

\subsection{Výkonostní požadavky}

Výkonostní požadavky určují parametry zátěže, na kterou má být aplikace navrhnuta.

\begin{itemize}
    \item Každý uživatel může mít desítky až stovky zdrojů.
	\item Zdroj může poskytovat desítky položek denně.
    \item Každý zdroj se kontroluje rozumně často.
    \item Aplikace bude dobře škálovatelná.
\end{itemize}

\section{Důsledky požadavků na architekturu}

Z požadavků vyplývají důsledky na volbu architektury aplikace.
Aplikce se bude skládat ze tří částí, které spolu komunikují.

\subsection{Serverová část}

Serverá část je zodpovědná za získávání nových položek z jednotlivých zdrojů, uložení všech dat včetně uživatelského nastavení a poskytování těchto dat klientských částem.

\subsection{Klientská část}

Klientská část zobrazuje data získaná ze serveru a komunikuje s uživatelem, který skrze ni aplikaci ovládá.

\subsection{Doplněk do prohlížeče}

Nepovinné rozšíření do nejběžnějších prohlížečů, které má za cíl zjednodušit práci s aplikací.
Toto rozšíření neposkytuje žádnou vyšší funkcionalitu, které by nešlo docílit klientskou částí, avšak umožňuje uživateli přidat movou položku do vého manuálního zdroje dvěma kliknutími myší.

\section{Systém doporučování zdrojů}

Abychom uživatelům nabídli možnost objevovat nové zdroje, které jsou podobné těm, jež sledují, bude aplikace obsahovat systém doporučování zdrojů na základě podobnosti.
Na tomto místě bychom chtěli popsat a rozebrat různé způsoby doporučování.


