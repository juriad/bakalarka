\chapter{Design}

V kapitole design popíšeme návrh realizace požadavků, které jsme popsali v kapitole analýza.

\section{Architektura aplikace}

Jako implementační jazyk jsme zvolili jazyk Java.

\subsection{Serverová část}

% TODO definice Cloudu
Serverová část aplikace poběží na serverech v Cloudu, to nám umožní efektivní správu prostředků, možnost trvalé expanze a škálovatelnost aplikace.
Pro volbu konkrétního poskytovatele jsme měli (v době zahájení práce) na výběr z:
\begin{itemize}
	\item Amazon EC2
	\item Google AppEngine
\end{itemize}

Jelikož jsme měli již zkušenost s druhým zmíněným a jeho architektura a API nám byla známá, zvolili jsme jej.

\subsubsection{Platforma Google AppEngine}

% TODO definice VPS
Google AppEngine je platforma pro vývoj a hostování webových aplikací psaných v jednom z několika podporovaných jazyků.
Poskytovatel účtuje jen prostředky skutečně využité, tím se liší od běžných VPS.
Platforma umožnuje automatické škálování výkonu podle aktuální potřeby aplikace.
Samotný výběr poskytovatele nám umožňuje splnit výkonostní požadavky.

AppEngine je platforma, která poskytuje několik spolu provázaných rozhraní.
Pro naši aplikaci jsou důležité především:
\begin{itemize}
	\item Datastore --- databáze pro uložení všech dat, jak sdílených, tak i dedikovaných jednotlivým uživatelům.
		Specifikem této databáze je, že formálně nemá schéma a má jen velice omezené možnosti dotazování na úkor výkonu.
		Mezi největší omezení patří nemožnost jakýchkoli spojení tabulek, veškeré filtry použité v dotazu mohou jen porovnávat hodnotu atributu s konstantou; toto omezení nutně ovlivnilo návrh entit a implementaci tříd.
		Dalším omezením je nutnost existence perfektního indexu pro každý dotaz, který se kdy bude provádět; neexistence indexu způsobí běhovou chybu.
		Na druhou stranu, nutnost zaindexování všech dotazů umožňuje extrémní rychlost databáze; každý dotaz se provádí ve dvou krocích:
		1. nalezení prvního výsledku v indexu; 2. výpis nalezeného záznamu a všech následujících, dokud záznam vyhovuje všem filtrům.
		Existují ještě další omezení, které nás příliš netrápí, například, že dotaz může vrátit neaktuální výsledky.
	\item Task Queue --- fronta úloh, skrze kterou projdou všechny úlohy.
		Fronta se stará o pravidelné spouštění úloh s pevnou frekvencí; to zajišťuje, že při kontrole zdrojů nedojde k většímu zatížení serveru.
		Kontrola každého zdroje je jedna úloha a fronta úloh se stará o to, aby bylo spuštěno jen určité množství úloh zároveň.
	\item Users --- ověřování uživatelů; aplikace přebírá údaje o aktuálně přihlášeném uživateli při každém požadavku.
		Nemusíme tedy řešit celou infrastrukturu kolem registrace, přihlašování, odhlašování, zapomenutého hesla.
	\item BlobStore a Files --- jelikož architektura serveru neumožňuje práci s filesystémem -- není známá informace o tom, na kterých serverech a kde na světě aplikace právě běží (potenciálně v několika instancích) -- posktuje server rozhraní ke své abstrakci souborů.
		My toto API použijeme k uložení a následné prezentaci zazálohovaných webových stránek článků.
\end{itemize}

% TODO link Slim3
Pro obalení nízkoúrovňového API pro přístup k databázi jsme použili framework Slim3.
Slim3 zajišťuje mapování databázových bezschématických entit na reálné javové objekty.
Tento framework zároveň poskytuje objektově konzistentní rozhraní ke skládání databázových dotazů.

\subsection{Klientská část}

% TODO vysvětlit html, css, js
Klientská část apliakce je ta část, kterou si každý uživatel stáhne při návštěvě webové stránky aplikace.
Jelikož se aplikace spouští v internetovém prohlížeči, je omezená tamním prostředím: grafická část je tvořena jazykem HTML a CSS, výkonná část je napsána v jazyce JavaScript.
Jelikož serverová část aplikace je vytvořena v jazyce Java, zvolili jsme pro klientskou část nástroj, který umožňuje zkompilovat kód v jazyce Java do jazyku JavaScript, který je určen pro běh v prohlížeči.
Takovým nástrojem je Google Web Toolkit -- GWT; mezi jeho základní vlastnosti patří:
\begin{itemize}
	\item serverová i klientská část je psána ve stejném jazyce,
	\item společný kód pro serverovou a klientskou část se napíše jen jednou (a přeloží dvakrát),
	\item poskytuje základ pro komunikaci mezi klientskou a serverovou částí, stará se o serializaci a deserializaci požadavků,
	\item stará se o minifikaci kódu odeslaného klientovi,
	\item poskytuje hotové základní komponenty pro tvorbu uživatelského rozhraní.
\end{itemize}

\section{Entity}

Entity jsou obecné struktury, se kterými aplikace pracuje.
Cílem této kapitoly je entity popsat a vysvětlit úlohu, kterou hrají v kontextu celé aplikace.

\subsection{Oblast uživatele a konfigurace}

\subsubsection{Uživatel}

Uživatel je reprezentace osoby, která systém používá.
Entitu uživatele přebíráme od poskytovatele serveru pomocí Users API.

\subsubsection{Konfigurační položka}

Konfigurační položka nemá nic společného s termínem \textit{položka} and entitou položka.

Jedna konfigurační položka definuje jeden parametr, jak má aplikace vypadat nebo jak se má chovat; lze rozlišit dva typy konfiguračních položek:
\begin{itemize}
	\item konfigurace serverové části --- závisí na ní chování serverové části,
	\item konfigurace klientské části --- závisí na ní vzhled nebo chování klientské části.
		Hodnoty položek tohoto typu mohou záviset na konkrétním uživateli, který aplikaci používá.
\end{itemize}

\subsection{Oblast zdrojů}

Tato část seskupuje všechny entity, které jsou souvisí se zdroji položek, jejich reprezentací a zpracováním.

\subsubsection{Zdroj}

Zdroj reprezentuje každou webovou stránku nebo službu, kterou aplikace používá k získávání nových položek.
Aplikace pravidelně kontroluje změny na zdrojích; pokud ke změně dojde, aplikace přidá odpovídající položky.
Zdroj je abstraktní entita, od které jsou odvozeny jednotlivé typy zdrojů; obsahuje všechny informace nutné k tomu, aby zdroj mohl být periodicky kontrolován.

\paragraph{Manuální zdroj}

Manuální zdroj patří vždy jednomu uživateli.
Nereprezentuje žádnou internetovou adresu a sám o sobě neposkytuje žádné položky.
Všechny položky, které tomuto zdroji patří, přidal sám uživatel skrze jednu ze služeb k tomu určených.

\paragraph{RSS/Atom}

RSS a Atom jsou formáty XML dokumentů, které popisují novinky, ke kterým došlo za poslední dobu na webovém portálu.
Tento zdroj typicky obsahuje odkazy na články, které byly publikovány v nedávné minulosti.

\paragraph{Webový rozcestník}

Nahrazuje funkcionalitu RSS či Atom pro webové portály, které tyto kanály neposkytují.
Vychází z jednoduchého pozorování, že úvodní stránka takového portálu obsahuje seznam nedávných článků a že položky převzaté z RSS či Atomu by odpovídaly odkazům na této stránce.

\paragraph{Změna stránky}

Reprezentuje webovu stránku o jejíž změně obsahu chce uživatel být informován.
Každá změna stránky způsobí vytvoření položky informující o změně.

\subsubsection{Region}

Pro typ zdroje Webový rozcestník a Změna stránky definuje oblast stránky, jež je z pohledu uživatele zajímavá.
Oblast na stránce je určena výběrem oblastí stránky, které se mají kontrolovat a oblastí, které se mají při kontrole ignorovat.
Motivací k pozitivnímu a negativnímu přístupu k jednotlivým částem je fakt, že stránky obsahují obsah, který se mění při každém dotazu (reklamy, informace o času generování stránky) nebo není zajímavý (hlavička a patička stránky, menu).

\subsubsection{Uživatelský zdroj}

Entita zdroj slouží výhradně pro účely periodické kontroly, proto zavedeme entitu uživatelský zdroj, jež realizuje vztah mezi uživateli a zdroji.
Několik uživatelů může sledovat ten samý zdroj a každý uživatel většinou sleduje více zdrojů.
Zároveň tato entita uchovává vlastní nastavení každého zdroje zvlášť pro každého uživatele.



\section{Rozhraní pro komunikaci klienta se serverem}

\section{Procesy}

\section{Sběr položek}

\section{Doporučování}
