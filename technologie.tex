\chapter{Použité technologie}

Jako implementační jazyk aplikace jsme zvolili jazyk Java.
Tento výběr ovlivňuje volbu jednotlivých knihoven a frameworků, které nám budou poskytovat a zprostředkovávat různé technologie.
Z povahy aplikace vyplývá, že většina netriviálních úloh bude probíhat na serverové části, tam také použijeme nejvíce knihoven.

\section{Poskytovatel platformy}

% TODO definice Cloudu
Serverová část aplikace poběží na serverech v Cloudu, to nám umožní efektivní správu prostředků, možnost trvalé expanze a škálovatelnost aplikace.
Pro volbu konkrétního poskytovatele jsme měli (v době zahájení práce) na výběr z následujících možností:
\begin{itemize}
	\item \zkratka{VPS}{Virtual Private Server -- virtualizovaný server, který si může kdokoli pronajmout.}
    \item Amazon EC2
    \item Google AppEngine
\end{itemize}

Virtuální server sice poskytuje největší svobodu, ale za cenu nutné vlastní konfigurace systému.
Tomu jsme se chtěli vyhnout a zaměřit své úsilí na samotný vývoj aplikace.
Další jeho nevýhodou je nemožnost automatického škálování.
Až na výjimky\footnote{Jedinou nám známou výjimkou je poskytovatel \url{http://4smart.cz/}} se účtuje paušální měsíční poplatek.

Zbylí dva poskytovatelé nabízí hotovou platformu, na níž je možné rovnou začít stavět aplikaci.
Jelikož jsme měli již zkušenost s posledním zmíněným a tedy jeho architektura nám byla známá, zvolili jsme jej.

\subsection{Platforma Google AppEngine}

Google AppEngine je platforma pro vývoj a hostování webových aplikací psaných v jednom z několika podporovaných jazyků.
Poskytovatel účtuje jen prostředky skutečně využité, tím se liší od běžných VPS.
Platforma umožnuje automatické škálování výkonu podle aktuální potřeby aplikace.
Samotný výběr poskytovatele nám umožňuje splnit výkonostní požadavky.

AppEngine je platforma, která poskytuje několik spolu provázaných rozhraní.
Pro naši aplikaci jsou důležité především:
\begin{itemize}
    \item Datastore -- databáze pro uložení všech dat, jak sdílených, tak i dedikovaných jednotlivým uživatelům.
        Specifikem této databáze je, že formálně nemá schéma a má jen velice omezené možnosti dotazování na úkor výkonu.
        Mezi největší omezení patří nemožnost jakýchkoli spojení tabulek, veškeré filtry použité v dotazu mohou jen porovnávat hodnotu atributu s konstantou; toto omezení nutně ovlivnilo návrh entit a implementaci tříd.
        Dalším omezením je nutnost existence perfektního indexu pro každý dotaz, který se kdy bude provádět; neexistence indexu způsobí běhovou chybu.
        Na druhou stranu, nutnost zaindexování všech dotazů umožňuje extrémní rychlost databáze; každý dotaz se provádí ve dvou krocích:
        1. nalezení prvního výsledku v indexu; 2. výpis nalezeného záznamu a všech následujících, dokud záznam vyhovuje všem filtrům.
        Existují ještě další omezení, které nás příliš netrápí, například, že dotaz může vrátit neaktuální výsledky.
    \item Task Queue -- fronta úloh, skrze kterou projdou všechny úlohy.
        Fronta se stará o pravidelné spouštění úloh s pevnou frekvencí; to zajišťuje, že při kontrole zdrojů nedojde k většímu zatížení serveru.
        Kontrola každého zdroje je jedna úloha a fronta úloh se stará o to, aby bylo spuštěno jen určité množství úloh zároveň.
    \item Users -- ověřování uživatelů; aplikace přebírá údaje o aktuálně přihlášeném uživateli při každém požadavku.
        Nemusíme tedy řešit celou infrastrukturu kolem registrace, přihlašování, odhlašování, zapomenutého hesla.
    \item BlobStore a Files -- jelikož architektura serveru neumožňuje práci s filesystémem -- není známá informace o tom, na kterých serverech a kde na světě aplikace právě běží (potenciálně v několika instancích) -- posktuje server rozhraní ke své abstrakci souborů.
        My toto API použijeme k uložení a následné prezentaci zazálohovaných webových stránek článků.
\end{itemize}

\subsection{Použité knihovny}

% TODO přeformulovat
Abychom znovu nevynalézali kolo a neřešili problémy, které již jiní řešili před námi, použijeme několik knihoven, které nám zjednodušší implementaci vlatní aplikace.

\subsubsection{Slim3}
% TODO link Slim3
Pro obalení nízkoúrovňového API pro přístup k databázi použijeme framework Slim3.
Slim3 zajišťuje mapování databázových bezschématických entit na reálné javové objekty.
Tento framework zároveň poskytuje objektově konzistentní rozhraní ke skládání databázových dotazů.
Tato knihovna je vyvinutá jen a pouze pro platformu AppEngine, neboť je silně závislá na Datastore API.

\subsubsection{ROME}
%TODO link ROME
ROME je knihovna, která poskytuje sadu parserů a generátorů pro různé typy formátů zpravujících o novinkách na webovém zdroji.
My tuto knihovnu použijeme pro oba směry konverze, využijeme parsery při kontrole jednoho typu zdrojů a generátory při sdílení (exportu) položek.
ROME je starší knihovna, nezdá se, že by její vývoj pokračoval, nicméně je dostatečně robustní a stabilní pro naše potřeby.

\subsubsection{Apache HttpComponents}
% TODO apache http components
% definice http
Apache HttpComponents je široce rozšířená kni\-hov\-na uřcená pro zajištění komunikace přes HTTP.
My ji použijeme pro stahování webových dokumentů při kontrole zdrojů všech typů (kromě manuálního).

\subsubsection{JSOUP}
% TODO vysvětlit DOM
% TODO link jsoup
Knihovna jsoup umožňuje rozparsovat HTML stránku do DOM struktury a poskytuje metody na procházení a manipulaci se stromem elementů.
Knihovna zvládá zpracování i takových stránek, které nevyhovující standardům.
My knihovnu využijeme ve dvou situacích:
\begin{itemize}
    \item pro parsování webové stránky při kontrole některých typů zdrojů,
    \item pro úpravu relativních odkazů při zálohování webové stránky.
\end{itemize}

\subsubsection{JScience}
% TODO link JScience
Knihovna JScience poskytuje implementaci nejrůznějších výpočetních algoritmů určených pro matematiku a fyziku.
Nás z této knihovny bude zajímat jen malá část: implementace vektorových a maticových operací, které použijeme při výpočtech doporučených zdrojů pro jednotlivé uživatele.

\subsubsection{Diff Match Patch}
% TODO link diff match patch
Diff match patch je malá knihovna, která nabízí robustní algoritmy pro operace nutné při synchronizaci textu.
My využijeme jen jednu ze tří částí: porovnání dvou textů, které nám vrátí seznam rozdílů.

\section{Klientská část}

% TODO vysvětlit html, css, js
% TODO přeformulovat výraz stáhne
Klientská část naší apliakce je ta část, kterou si každý uživatel stáhne při návštěvě webové stránky aplikace.
Jelikož se aplikace spouští v internetovém prohlížeči, je omezená tímto prostředím: grafická část je tvořena jazykem HTML a CSS, výkonná část je napsána v jazyce JavaScript.
Jelikož serverová část aplikace je vytvořena v jazyce Java, zvolili jsme pro klientskou část nástroj, který umožňuje zkompilovat kód v jazyce Java do jazyku JavaScript, který je určen pro běh v prohlížeči.
Takovým nástrojem je Google Web Toolkit -- GWT; mezi jeho základní vlastnosti patří:
\begin{itemize}
    \item serverová i klientská část je psána ve stejném jazyce,
    \item společný kód pro serverovou a klientskou část se napíše jen jednou (a přeloží dvakrát),
    \item poskytuje základ pro komunikaci mezi klientskou a serverovou částí, stará se o serializaci a deserializaci požadavků,
    \item stará se o minifikaci kódu odeslaného klientovi,
    \item poskytuje hotové základní komponenty pro tvorbu uživatelského rozhraní.
\end{itemize}

\subsection{Použité knihovny}

V klientské části aplikace použijeme jedinou knihovnu: GWT Eye Candy.
Tato knihovna nabízí kolekci ovládacich prvků do uživatelského prostředí aplikace.
My využijeme jen jediný prvek: ColorPicker pro výběr barvy štítků.

\section{Doplněk do prohlížeče}

% TODO link na crossrider
Využijeme služeb platformy Crossrider, která umožňuje vytvořit jediný doplněk, který bude fungovat shodně v nejrozšířenějších prohlížečích.
Podporované prohlížeče jsou:
% TODO link na FAQ 
\begin{itemize}
    \item Internet Explorer 7 a vyšší
    \item Chrome -- všechny verze
    \item Firefox -- 3.6 a vyšší
\end{itemize}

\subsection{Použité knihovny}

% TODO knihovny v doplňku

