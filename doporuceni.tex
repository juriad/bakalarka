\chapter{Systém doporučování zdrojů}

Abychom uživatelům nabídli možnost objevovat nové zdroje, které jsou podobné těm, které již sledují, bude aplikace obsahovat systém doporučování zdrojů na základě podobnosti.
Na tomto místě popíšeme a rozebereme různé způsoby doporučování.
Ve formulaci problematiky budeme používat pojmy uživatel -- osoba, která interaguje se systémem a položka -- objekt, který nabízíme a chceme doporučovat; v našem případě je položkou ve formulaci problematiky jakýkoli zdroj, který uživatelé sledují.

\section{Globální doporučování}

Nejjednodušším způsobem doporučování je využití globálních statistik o všech položkách.
Vychází z předpokladu, že existují položky, které jsou objektivně kvalitní a vhodné pro většinu uživatelů.
Kritérii, podle kterých jsou položky seřazeny, mohou být:
\begin{itemize}
    \item počet uživatelů, kteří si zakoupili položku,
    \item počet zhlédnutí položky,
    \item průměr hodnocení všech uživatelů,
	\item počet recenzí položky a další.
\end{itemize}
Tento způsob používají některé menší internetové obchody z důvodu snadné implementace a nenáročnosti výpočtu.

Globální doporučování je v našem případě nevhodné, naším cílem není doporučovat velké zpravodajské servery, které zná každý uživatel.
Naším cílem je naopak nabídnout uživatelům i menší zdroje, které třeba nemusí být mezi uživateli příliš známé.
Takové zdroje mohou být úzce zaměřené, ale přesto vysoce kvalitní; příkladem jsou blogy různých vývojářů.

\section{Doporučování na základě podobnosti}

Všechny systémy doporučování vyžadují prvotní znalost o uživateli -- modelu, který popisuje preference uživatele.
Model uživatele může obsahovat nejrůznější informace:
\begin{itemize}
    \item údaje, které si o sobě vyplní sám uživatel: oblíbený žánr (u hudby), oblíbený autor (u knih),
    \item hodnocení, které sám vyplnil: klikl na tlačítko \uv{Líbí/nelíbí}, ohodnotil položku na stupnici 1--10,
    \item záznam dosavadní interakce se systémem: písničky, jež si uživatel poslechl; knihy, které zakoupil,
	\item případně podrobněji: kategorie či produkty, které navštívil; čas, který strávil na jednotlivých stránkách.
\end{itemize}

Existují dva základní přístupy k nabízení položek na základě podobnosti.
Buď doporučují položky na základě podobnosti obsahu -- detailní znalosti atributů -- položek, které se uživateli již líbily,
nebo naopak porovnávají, jaký vztah k položce mají podobní uživatelé.
Nároky na systémy doporučování ve velkých obchodech si vynucují konstrukci hybridních algoritmů, které se snaží získat dobré vlastnosti z obou skupin.

\subsection{Doporučování založené na podobnosti obsahu}

V případě, že položky v systému jsou takové povahy, že jsou o nich známé podrobné informace, může systém nabízet položky, které jsou podle nějakých kritérií blízké těm, které se uživateli již dříve líbily.
Potřebnými podrobnými informacemi mohou být například~\cite{pazzani2007content}:
\begin{itemize}
    \item zvukové charakteristiky hudební nahrávky, její tempo, melodie apod.,
    \item text jednotlivých knižních děl jako vektory ve vektorovém prostoru nad slovy.
\end{itemize}

Tento přístup však není v našem případě použitelný, nemáme dostatečné informace o jednotlivých zdrojích; jediné, co známe je typ zdroje a jeho internetová adresa.

\subsection{Doporučování na základě podobnosti uživatelů}

V dobách, kdy ještě neexistovaly počítače, si lidé mezi sebou doporučovali ku příkladu knihy, které se jim líbily.
Jelikož se obvykle přátelí lidé, kteří mají podobné zájmy, je vysoce pravděpodobné, že kniha, která se jedné osobě líbila, se bude líbit i jejím přátelům.

Doporučování založené na podobnosti uživatelů je na tomto principu založené~\cite{schafer2007collaborative}.
Pokud budeme mít informace o oblíbených položkách mnoha uživatelů, můžeme se uživateli pokusit doporučit takové, které se líbí jeho okolí.
Okolím myslíme v tomto kontextu všechny uživatele, kteří jsou jemu podobní.

Všechny algoritmy tohoto typu pracují s maticí uživatelů a položek; v řádcích jsou všichni uživatelé, ve sloupcích jsou všechny položky.
Prvek matice vyjadřuje hodnocení $i$-tého uživatele $j$-té položky.
Pro nalezení vhodných položek k doporučení můžeme porovnávat vektory:
\begin{itemize}
    \item uživatelů pro všechny položky: pokud jsou si dva vektory blízké, líbí se položka stejné skupině uživatelů,
    \item položek pro všechny uživatele: pokud jsou si dva vektory blízké, hodnotí uživatelé položky podobně -- mají stejný vkus.
        Na základě omezené skupiny podobných uživatelů se v druhém kroku dopočítají konkrétní položky k doporučení.
\end{itemize}

Podobnost vektorů je možné porovnávat několika způsoby, například mírou jejich korelace či skalárním součinem (úhel mezi vektory).

\section{Volba systému doporučování}\label{sec:volba-systemu-doporucovani}

Chování dat v naší aplikaci se liší od běžných modelů doporučování; mezi specifické vlastnosti patří:

\begin{itemize}
    \item Uživatelů bude relativně málo; doporučení se vypočítává pouze uživatelům, kteří aplikaci používají (sledují alespoň jeden zdroj).
    \item Změna hodnocení položek (v našem případě: 0 -- uživatel zdroj nesleduje, 1 -- uživatel zdroj sleduje) neprobíhá příliš často.
        Vycházíme z pozorování, že uživatel stále sleduje stejné zdroje.
        Bude nám tedy stačit přepočítávat doporučení offline -- například jednou za den, místo na prostředky náročného výpočtu při každém dotazu.
    \item Matice uživatelů a zdrojů je řídká; mohou sice existovat zdroje, které sleduje většina uživatelů, ale žádný uživatel nebude sledovat většinu zdrojů.
        Mnoho zdrojů bude sledováno právě jedním uživatelem, půjde nejčastěji o různé specificky zaměřené blogy.
    \item Nemáme téměř žádné využitelné informace o samotných zdrojích, jen jejich typ a internetovou adresu.
\end{itemize}

Z výše uvedených předpokladů vyplývá, že nejvhodnější volbou algoritmu bude některá z variant doporučování založená na podobnosti uživatelů.
Jako největší problém při návrhu doporučování v naší aplikaci považujeme řídkost dat: obáváme se, že uživatelé nebudou dostatečně sdílet své zdroje, aby doporučení dávalo relevantní výsledky.

Zvolili jsme algoritmus, který navrhli Z. Huang, D. Zeng, H. Chen ve svém článku~\cite{huang2004link}.
Tento algoritmus je do velké míry inspirovaný algoritmy známými z prostředí internetových vyhledávačů, jako je například algoritmus PageRank~\cite{page1999pagerank}.

\subsection{Popis algoritmu}

Algoritmus si v průběhu výpočtu udržuje seznam zdrojů, které jsou vhodné k doporučení uživateli.
Princip algoritmu je následující:
\begin{enumerate}
    \item inicializuje seznam zdrojů našeho uživatele zdroji, které již sleduje,
    \item najde všechny uživatele, kteří sledují některý ze zdrojů, které jsou v seznamu našeho uživatele: nazveme je sousedy,
    \item vypočítá, jak jsou si náš uživatel a soused blízcí, podle toho přidělí každému sousedovi váhu,
    \item přidá všechny zdroje všech sousedů do seznamu zdrojů našeho uživatele a přiřadí jim váhy podle toho, kteří sousedé je sledují,
    \item iteruje několikrát kroky 2--4,
    \item vrátí nový seznam zdrojů vhodných pro našeho uživatele seřazený podle jejich váhy.
\end{enumerate}

Výhody a nevýhody tohoto algoritmu:
\def\proitem{\item[$\boldsymbol{+}$]}
\def\conitem{\item[$\boldsymbol{-}$]}
\begin{itemize}
    \proitem umožňuje vypočítat doporučení pro všechny uživatele najednou,
    \proitem všechny operace lze implementovat efektivně jako operace s maticemi; kroky 2--4 spočívají pouze v násobení matic,
	\conitem algoritmus je iterativní -- iterace by ideálně měla probíhat, dokud se výsledek neustabilizuje,
	\conitem je nutné udržovat v paměti celou matici, přestože nás může zajímat doporučení pro jediného uživatele.
\end{itemize}

\subsection{Příklad doporučení}

Naše aplikace bude vypočítávat doporučení pro všechny uživatele jednou za 24 hodin.
Výsledky doporučení budou uživatelům dostupné v dialogu pro přidání nového zdroje.
Mohou si vybrat, zda zadají vlastní internetovou adresu zdroje nebo chtějí sledovat některý z doporučených zdrojů.

\begin{figure}
    \centering
    \includegraphics[width=14.5cm]{doporuceni.eps}
    \caption{Uživatelé sledující zdroje a jejich doporučení}
    \label{fig:doporuceni}
\end{figure}

Abychom přiblížili možnosti algoritmu, uvedeme příklad doporučení zdrojů na příbězích pěti lidí\footnote{Popsané osoby jsou vymyšlené; uvedené zdroje skutečně existují, jsou jimi:
\href{http://bugemos.com/}{bugemos.com}, \href{http://technet.idnes.cz/}{technet.idnes.cz},
\href{http://osel.cz/}{osel.cz}, \href{http://xkcd.com/}{xkcd.com}, 
\href{http://www.abclinuxu.cz/}{abclinuxu.cz}, \href{http://www.root.cz/}{root.cz},
\href{http://www.zdrojak.cz/}{zdrojak.cz}, \href{http://webylon.info/}{webylon.info} a
\href{http://interval.cz/}{interval.cz}.
}.

\begin{itemize}
	\item Věra je běžná uživatelka, která sleduje zpravodajský magazín o technice (technet) a ráda se pobaví u vtipného komixu (bugemos).
	\item Konstantin se také zajímá o techniku, stejně jako Věra sleduje technet.
		O téma se však zajímá hlouběji, čte články na oslovi a baví se náročnějším komixem (xkcd).
	\item Rút je srdcem linuxačka; mezi její sledované weby patří především abclinuxu a root.
		Zajímá se okrajově o techniku (technet) a čte komix xkcd.
	\item Linda je Rútina kamarádka; začíná s tvorbou webu, sleduje tedy návody na zdrojáku.
		Ve čtečce má i web root, protože chce vědět, co je ten Linux zač.
	\item Waldemar se věnuje tvorbě webových stránek, a proto sleduje všechny tématické weby (zdroják, webylon, interval).
		Jako mnoho ostatních má rád komixy na xkcd.
\end{itemize}

Na schématu~\ref{fig:doporuceni} je znázorněno, který uživatel sleduje které zdroje.
Nad každým uživatelem je seznam zdrojů, které mu byly doporučeny.

\begin{vypis}
	\verbatiminput{doporuceni.out}
	\caption{Výstup algoritmu doporučení po 10 iteracích.}
	\label{vypis:doporuceni}
\end{vypis}

Doporučení jsme vypočítali skriptem, který je dostupný na přiloženém CD a přesně odpovídá implementaci, kterou používame ve vlastní aplikaci.
Jeho výstupem (výpis~\ref{vypis:doporuceni}) je trojice řádků pro každého uživatele: jméno; sledované zdroje; doporučené zdroje (jejich váha).

\begin{itemize}
	\item Věře jsou doporučeny zdroje, které sledují její sousedé (Konstantin a Rút).
		Přednost získaly zdroje, které sledují jen sousedé (osel, abclinuxu), protože se netříští jejich síla (xkcd a root sledují i jiní uživatel, kteří jsou Věře vzdálenější).
		Za dobré doporučení můžeme považovat zdroje osel a xkcd.
	\item Konstantinovi jsou doporučeny opět zdroje jeho sousedů (bugemos, abclinuxu) a root, u kterého se opět dělí jeho síla mezi více uživatelů.
		Dobrým doporučením je jen bugemos.
	\item Rútina záliba v komixech a technice jí způsobí doporučení osla (s Konstantinem, který ho sleduje má společnou polovinu zdrojů).
		Rút je si s Konstaninem velice blízká, to je důvodem zvýšení váhy technetu (který sledují oba).
		Tedy systém doporučení preferuje Věru před Lindou a doporučí bugemos před zdrojákem.
		Bugemos a osel jsou doporučení.
	\item Lindě jsou doporučeny zdroje jejích sousedů (Waldemara a Rút).
		Waldemarovi zdroje mají větší váhu, protože Waldemar získal doporučením dvou zdrojů větší váhu (projev iterace algoritmu).
	\item Waldemarovi jsou doporučeny zdroje, které sledují sousedé (Linda, Rút, Konstantin). Abclinuxu a osel mají větší váhu než technet (netříští se jejich síla; $1 > \tfrac{2}{3}$).
		Nejvhodnějším zdrojem pro Waldemara je ale root, který je sledovaný všemi jeho sousedy.
		Waldemarovi není doporučen žádný vhodný zdroj (bugemos, který by jej mohl zajímat, je v grafu příliš daleho).
\end{itemize}
