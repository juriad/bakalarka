%%% Strana s čestným prohlášením k bakalářské práci

\newpage

\vglue 0pt plus 1fill

\noindent
Prohlašuji, že jsem tuto bakalářskou práci vypracoval samostatně a výhradně
s~použitím citovaných pramenů, literatury a dalších odborných zdrojů.

\medskip\noindent
Beru na~vědomí, že se na moji práci vztahují práva a povinnosti vyplývající
ze zákona č. 121/2000 Sb., autorského zákona v~platném znění, zejména skutečnost,
že Univerzita Karlova v Praze má právo na~uzavření licenční smlouvy o~užití této
práce jako školního díla podle §60 odst. 1 autorského zákona.

\vspace{10mm}

\hbox{\hbox to 0.5\hsize{%
V ........ dne ............
\hss}\hbox to 0.5\hsize{%
Podpis autora
\hss}}

\vspace{20mm}
\newpage

%%% Povinná informační strana bakalářské práce

\vbox to 0.5\vsize{
\setlength\parindent{0mm}
\setlength\parskip{5mm}

Název práce:
Webová čtečka a knihovna článků
% přesně dle zadání

Autor: % jméno
Adam Juraszek

Katedra:  % Případně Ústav:
Katedra softwarového inženýrství
% dle Organizační struktury MFF UK

Vedoucí bakalářské práce:
RNDr. Martin Svoboda, Katedra softwarového inženýrství
% dle Organizační struktury MFF UK, případně plný název pracoviště mimo MFF UK

Abstrakt:
Cílem práce je navrhnout a implementovat čtečku jako webovou aplikaci, která umožňuje sledovat zdroje článků na Internetu prostřednictvím kanálů RSS a Atom.
Od podobných aplikací se odlišuje integrací dalších služeb a možnostmi: sledování stránek, které tyto kanály neposkytují, zálohování stránek, ukládání článků do uživatelem spravované knihovny a systémem doporučování zdrojů.
Aplikace je naprogramovaná v jazyku Java a skládá se ze tří částí: serverové části vytvořené na platformě Google AppEngine, klientské části využívající framework GWT a volitelného doplňku do běžných prohlížečů.
Webová čtečka je navržená s ohledem na práci s velkým množstvím dat a lze ji snadno rozšířovat.
Práce rovněž analyzuje charakteristiky dostupných čteček a rozebírá různé systémy doporučování.

% abstrakt v rozsahu 80-200 slov; nejedná se však o opis zadání bakalářské práce

Klíčová slova:
webová aplikace, RSS čtečka, AppEngine, GWT
% 3 až 5 klíčových slov


\vss}\nobreak\vbox to 0.49\vsize{
\setlength\parindent{0mm}
\setlength\parskip{5mm}

Title: 
Web Reader and Article Library
% přesný překlad názvu práce v angličtině

Author:
Adam Juraszek

Department: % katedra
Department of Software Engineering
% dle Organizační struktury MFF UK v angličtině


Supervisor: % vedouci
RNDr. Martin Svoboda, Department of Software Engineering
% dle Organizační struktury MFF UK, případně plný název pracoviště
% mimo MFF UK v angličtině

Abstract:
The aim of this thesis is to design and implement a reader web application, which can watch article sources on the Internet using RSS and Atom feeds.
The difference from similar applications is in integration of other services and options: watching websites that do not provide news feeds, backing up web pages, saving articles into user-managed library and a source recommendation system.
The application is programmed in Java and consists of three parts: the server part developed on Google AppEngine, the client part using GWT framework and an optional cross-browser extension.
The Web reader is designed to work with large amounts of data and can be easily extended.
The thesis also contains an analysis of the characteristics of available readers and various recommendation systems.
% abstrakt v rozsahu 80-200 slov v angličtině; nejedná se však o překlad
% zadání bakalářské práce

Keywords:
web application, RSS reader, AppEngine, GWT
% 3 až 5 klíčových slov v angličtině

\vss}


