\chaptert{Závěr}

Výsledkem naší práce na tomto projektu je aplikace, která je, dle našeho názoru, schopná nabídnout uživatelům vlastnosti, které se od čtečky očekávají.
Aplikace v sobě navíc kombinuje několik dalších služeb, díky nimž je unikátní; nejedná se \uv{jen o další čtečku}.
Z tohoto pohledu považujeme náš cíl za splněný.

\sectiont{Vlastnosti čtečky}

Čtečku jsme navrhli a vytvořili především tak, aby odpovídala našim požadavkům.
Aplikace nabízí následující možnosti:
\begin{itemize}
	\item sledovat libovolný kanál RSS nebo Atom,
	\item získávat informace o změnách i ze stránek, které zmíněné kanály neposkytují,
	\item doporučovat zdroje, které sledují podobní uživatele,
	\item přidat vlastní internetovou adresu jako novou položku (použitím rozhraní aplikace nebo prostřednictvím volitelného doplňku),
	\item přiřadit položkám štítky,
	\item filtrovat položky podle zdrojů či štítků,
	\item ukládat své filtry a používat je jako seznamy položek,
	\item zálohovat aktualní stav webové stránky reprezentované položkou,
	\item exportovat seznam položek jako RSS nebo Atom kanál,
	\item ovládat většinu aplikace jen pomocí konfigurovatelných klávesových zkratek.
\end{itemize}

\sectiont{Možná rozšíření}

Vzhledem k šíři problematiky, kterou se naše aplikace snaží pokrýt, je zřejmé, že nemůže v této verzi obsahovat veškeré vlastnosti a funkce, které by nás mohly dále napadnout.
Aplikace srovnatelného rozsahu vytvářejí celé týmy zkušených vývojářů.
Snažili jsme se proto především vytvořit pevné základy, na kterých bude možné dále pokračovat.

Uvedeme zde některá témata rozšíření, která by možnosti naší aplikace mohly výrazně posunout dopředu.

\subsectiont{Import a export zdrojů}
Import by umožnil snazší příchod nových uživatelů.
Již existuje standardizovaný formát \zkratka{OPML}{Outline Processor Markup Language -- XML formát popisující stromové struktury}, který používají všechny běžné čtečky k usnadnění příchodu nového uživatele či odchodu stávajícího.

\subsectiont{Pravidla přiřazování štítků}
Jde o rozšíření současného systému; nyní může uživatel nastavit štítky přiřazované ke každé položce některého zdroje.
Uživatel by měl možnost vytvoření pravidel, která by rozhodovaly o automatickém přiřazení štítku při vytváření položky.

\subsectiont{Přizpůsobení uživatelského rozhraní}
Umožnilo by to používat naší aplikaci na mobilních telefonech a tabletech.
Podle velikosti a typu zařízení se zvolí vhodné rozložení uživatelského rozhraní aplikace.
 
\subsectiont{Nastavení vzhledu a chování štítku}
Pokud umožníme nastavit štítek tak, aby byl zobrazený, i když není přiřazený (odlišným stylem) a přidáme možnost nahrazení textu štítku ikonkou, bylo by možné docílit funkcionality známé z konkurenčních čteček: přepínání mezi prázdnou a plnou hvězdičkou.

\subsectiont{Klávesová makra}
Pokud by bylo možné provést více akcí se štítky najednou, šlo by například přepínat mezi několika štítky jedinou klávesou.
Mohli bychom tím simulovat prioritu nebo důležitost položky.

\subsectiont{Stav přečtení}
Během návrhu aplikace jsme zvolili reprezentaci stavu přečtení uživatelské položky jako zvláštního atributu.
Vzhledem k obecnosti dotazů, které do databáze pokládáme, je možné tento stav reprezentovat lépe dvojicí štítků \uv{přečtená} a \uv{nepřečtená}.
Výhody plynoucí ze změny by byly dvě: samotné zjednodušení modelu, poskytnutí více možností při definici seznamových filtrů.

\sectiont{Problémy při implementaci}

Během implementace jsme se potýkali s několika problémy, které bychom zde chtěli krátce zmínit.

\subsectiont{Kurzory dotazů do databáze}

Kurzor je ukazatel na poslední záznam seznamu výsledků databázového dotazu.
Chtěli jsme jej využít při realizaci nekonečného seznamu v situaci, když uživatel odroluje k jeho konci.
Tehdy se dotazujeme na uživatelské položky v databázi, které následují po položkách získaných minulým dotazem.

Použitá databáze sice umožňuje kurzory při dotazování, ale jejich omezení vylučuje použití v naší aplikaci.
Fungují jen v případě, že dotaz neobsahuje operátor \verb|OR|, ani operátor výčtu \verb|IN|; dotazy, jež pokládáme však tyto operátory obsahují. 

Naším řešením je využít uspořádání na klíčích položek: novější uživatelské položce přiřadíme vyšší hodnotu klíče, resp. přiřadí ji databáze z rostoucí sekvence.
Každý náš dotaz bude obsahovat podmínku ostré nerovnosti klíče v databázi a klíče poslední uživatelské položky získané minulým dotazem.

\subsectiont{Nekonečný seznam}

Nekonečného seznamu se týká i druhý problém, se kterým jsme se potýkali.
Podle přehledu grafických komponent, které knihovna GWT poskytuje, jsme usoudili, že nekonečný seznam můžeme realizovat komponentou \verb|CellList|.
Neuvědomili jsme si však plně důsledky její implementace -- buňka seznamu není widget, není možné uvnitř buňky mít aktivní prvky, může obsahovat jen prostý kus HTML kódu.
Náš návrh ale vyžaduje možnost rozbalení a skrytí obsahu uživatelské položky, zobrazení dynamického seznamu štítků.

Z důvodu tohoto omezení jsme museli naprogramovat svoji vlastní komponentu prakticky zcela od začátku.
Jelikož jsme v té době ještě neměli dostatečné znalosti GWT, strávili jsme tím hodně času, postup naší práce se téměř zastavil.
Další komponenty, které jsme museli vyvinout, jsme již zvládli vytvořit bez problémů; jednou z nich byla například komponenta nabízející výběr štítku ze seznamu.

GWT nás svojí nabídkou grafických komponent zklamalo, očekávali jsme jich více a kvalitnějších.
Několikrát jsme objevili chyby a byli jsme nuceni hledat řešení na fórech.
Často byl problém způsoben podle nás nedostatečným odstíněním programového rozhraní od aspektů webových stránek.

\sectiont{Obsah přiloženého CD}

Na přiloženém kompaktním disku je k dispozici jak uživatelská a programátorská dokumentace, tak i samotné zdrojové kódy aplikace a doplňku do prohlížeče.

FIXME: rozepsat obsah CD během víkendu

\begin{itemize}
	\item aplikace
	\item doplněk
	\item uživatelská dokumentace
	\item programátorská dokumentace
	\item instalační příručka
	\item skript demonstrující algoritmus doporučení
	\item ???
\end{itemize}
