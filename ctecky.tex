\chapter{Webové čtečky}

Na internetu je dnes dostupné poměrně velké množství nejrůznějšího softwaru a nejrůznějších služeb, které poskytují svým uživatelům přehledný výtah z článků, které se objeví na několika webových portálech.
Takové aplikace bývají často zaměřeny na různé cílové skupiny: některé se zaměřují na uživatele, kteří se chtějí o novém článku dozvědět okamžitě, jakmile je publikován, jiné zaujmou spíše uživatele, kteří mají příliš mnoho zdrojů na to, aby bylo v jejich silách je pravidelně navštěvovat; v neposlední řadě existují aplikace, které se snaží maximálně zvýšit komfort čtení článků svým uživatelům.

V této kapitole se pokusíme rozebrat charakteristiky jednotlivých typů aplikací.

\section{Typy čteček a jejich uživatelů}

Před vlastním dělením aplikací vyčleníme ty, které obsahují čtečku RSS jen jako doplněk nebo ji využívají jen jako informační kanál o svých změnách.
Mezi takové aplikace můžeme zařadit například multimediální centrum \projekt{XBMC}{http://xbmc.org/}, které obsahuje na úvodní obrazovce ve spodní části proužek s běžícím textem: tituly článků na domovské stránce projektu.
Přestože XBMC má editor RSS kanálů a je možné nastavit vlastní kanál, nemůžeme aplikace takového typu považovat za plnohodnotné RSS čtečky.

Touto a podobnými aplikacemi se zabývat nebudeme; nebudou nás zajímat ani služby, které sice umožňují sledovat RSS, ale není to jejich primárním cílem.
Příkladem takové služby může být titulní stránka vyhledávače \projekt{Seznam.cz}{http://seznam.cz/}, kterou si uživatelé mohou upravit a přidat svoje RSS kanály, jejichž položky se na stránce budou zobrazovat.
Další podobnou službou je \projekt{iGoogle}{http://www.google.com/ig}, který bude ukončený k 1.~listopadu 2013.

\bigskip

V následujících podkapitolách rozdělíme čtečky podle několika kritérií:
\begin{itemize}
	\item způsobu integrace do uživatelského prostředí,
	\item uživatele, na kterého jsou zaměřené.
\end{itemize}

Bude zřejmé, že výhody jednoho typu čteček jsou často nevýhodami jiného typu.

\subsection{Nativní aplikace}

Pro používání tohoto typu čteček je nutné nainstalovat nativní aplikaci do operačního systému.
S tím souvisí několik omezení, se kterými musí uživatel předem počítat:
\begin{itemize}
	\item výběr aplikací může být omezený operačním systémem, který uživatel používá,
	\item v některém prostředí nemusí mít uživatel oprávnění instalovat aplikace (práce, škola),
	\item aplikace musí být trvale spuštěná, neboť periodicky kontroluje zdroje jednotlivých kanálů.
\end{itemize}

Mezi nativní aplikace budeme řadit i samotný internetový prohlížeč, pokud nabízí možnost sledování RSS kanálů.

\subsubsection{Výhody}

Mezi výhody nativní aplikace patří:
\begin{itemize}
    \item rychlejší odezva.
        Aplikace může ukládat všechny položky na pevný disk, čímž odbourává nutnost neustálé komunikace a opakovaného stahování veškerých dat ze serveru poskytovatele služby.
    \item offline čtení.
        Aplikace může stáhnout kompletní webovou stránku, na kterou se uživatel může podívat a přečíst originální text i v době, kdy nemá přístup k internetu.
		Toho mohou využívat třeba uživatelé, kteří často cestují a chtějí si zkrátit čtením dlouhé čekání.
    \item nezávislost na jiném subjektu.
        Uživatel není závislý na provozovateli služby; webové služby mohou skončit nebo mít výpadek.
        U nativní aplikace maximálně hrozí zastavení jejího dalšího vývoje, což zpravidla nemívá zásadní vliv na její dostupnost a funkcionalitu.
	\item úplná kontrola nad daty.
		Někteří uživatelé nechtějí svěřit své údaje poskytovateli služby; informace o tom, které zdroje sledují či které články se jim líbí považuje za příliš důvěrné.
		Jiný druh uživatelů se může chtít aktivně zapojit do vývoje aplikace, přiblížit ji svým přáním.
\end{itemize}

\subsection{Webová služba}

Přesným opakem nativních aplikací jsou webové služby (webové aplikace); pro používání webové služby není vyžadována instalace aplikace, data jsou dostupná na kterémkoli zařízení s internetem.
Největší slabinou těchto čteček je však nutnost neustálého připojení na internet.

\subsubsection{Výhody}

Mezi výhody aplikací postavených na využití webových služeb patří:
\begin{itemize}
    \item synchronizace stavu mezi více počítači.
        Uživatel má přístup ke svým sledovaným zdrojům a odkazům na články odkudkoli; může se přesouvat mezi několika zařízeními a pokračovat v procházení seznamu článků.
    \item menší riziko přetížení zdroje.
        V případě, že jeden zdroj sleduje velké množství uživatelů, stačí jeden jediný dotaz pro stažení RSS dokumentu a uspokojení všech uživatelů najednou.
        S možností přetížení můžou mít problém nativní aplikace.
		Do této situace se dostal například poskytoval počasí, z jehož serverů si aplikace XBMC stahovala informace příliš často (každých 30 minut)~\cite{xbmc-weather}.
    \item není nutné přepínat mezi aplikacemi.
        Pokud uživatel prochází seznam člán\-ků na stránce webové služby, nemusí se pro zobrazení originálního článku přepínat do internetového prohlížeče.
        Toto omezení neplatí pro samotné internetové prohlížeče a je redukované pro některé nativní aplikace, které umožňují zobrazit zjednodušenou webovou stránku přímo uvnitř sebe.
    \item uživatel má dostupné všechny položky bez ohledu na četnost spuštění počítače.
        Uživatel nemusí být stále online, webová služba kontroluje zdroj RSS bez ohledu na stav svých uživatelů.
        Pokud uživatel nativní aplikace nespustí čtečku po dobu několika dnů, může ztratit informace o některých článcích, zejména u frekventovaných zdrojů.
\end{itemize}

\subsection{Běžný uživatel}

Aby čtečku běžní uživatelé používali, musí jim být schopná nabídnout nějaký přínos, zvýšené pohodlí oproti procházení několika webových stránek.
Čtečky, které se zaměřují na běžné uživatele spolu soupeří komfortem, který uživatelům nabízí.
Výpis seznamu článků může vypadat třeba jako magazín nebo naopak, čtečka může prezentovat každý článek jen velkým obrázkem.
Tyto čtečky většinou umožňují sledovat trendy, sledovat ostatní uživatele a kategorizovat obsah.

Obecně můžeme říci, že řeší potřebu uživatele, kterou bychom popsali následovně:
\uv{Mám chvilku volného času, tak se podívám, co je nového.
Než abych procházel několik webových stránek, tak navštívím svoji čtečku a zjistím, co je nového zajímavého.
Čtečka zajistí, že uvidím souhrn ze všech důležitých webů.}

Pokud má čtečka napojení na sociální sítě či sama je sociální sítí, může být uživatelova motivace ještě rozšířena:
\uv{Vím, že uvidím jen takové články, které se líbily ostatním, takže se mi pravděpodobně budou líbit také a nebudu se muset zabývat brakem.}

Tento přístup funguje uživatelům, kteří čtou články na internetu z důvodu, že se chtějí pobavit.
Na druhou stranu, pokud by se chtěli dozvědět nepopularizované, originální informace třeba z oblasti vědy a výzkumu, tyto čtečky jim nejspíše nebudou sloužit dobře.

\subsection{Náročný uživatel}\label{ssec:narocny-uzivatel}

Za náročného uživatele budeme považovat uživatele, který používá čtečku z důvodu, že sám již není schopný navštěvovat pravidelně stránky svých informačních zdrojů.
Důvodem může být sledování příliš mnoha zdrojů, na kterých je publikováno mnoho článků každý den.
\uv{Mnoho} v tomto kontextu znamená desítky zdrojů a stovky článků či zpráv denně.

Náročný uživatel klade na čtečku specifické nároky:
\begin{itemize}
    \item vyžaduje kompaktní zobrazení, nesmí se plýtvat místem.
        Obrazovka musí být vyhrazená seznamu položek; každá položka musí zabírat minimální prostor, aby se jich vešlo na obrazovku co nejvíce.
    \item preferuje prostý text před grafickými efekty; nic nesmí rozptylovat jeho pozornost.
    \item vyžaduje aplikaci, která bude zcela či z velké většiny ovladatelná klávesovými zkratkami; pohyb a klikání myši je pomalé.
        Náročný uživatel ve čtečce tráví spoustu času, který musí využít efektivně.
    \item nepotřebuje zjednodušené prostředí, protože investice do podrobné konfigurace se mu časem vyplatí.
\end{itemize}

\section{Přehled čteček}

Jak jsme již na začátku kapitoly uvedli, na trhu je velké množství čteček.
V této části krátce představíme některé, dle našeho názoru, zajímavé čtečky.
U každé rozebereme nabízené vlastnosti a popíšeme ji z hlediska kritérií, které jsme si dříve definovali.
Pokusíme se nalézt její klady a zápory a doporučit ji cílové skupině.

\subsection{RSSOwl}

\projekt{RSSOwl}{http://www.rssowl.org/} je zástupcem čteček, které si uživatel musí nainstalovat do svého operačního systému; mezi nimi RSSOwl vyniká nezávislostí na konkrétní platformě.
K plnému využití jejích funkcí je však nutné mít tuto aplikaci neustále spuštěnou; aplikace to uživateli vynahradí upozorňováním na nové položky.

Vzhledem k délce vývoje (RSSOwl je velice stará čtečka, její vývoj začal v roce 2003) a způsobu vývoje (je open-source, kdokoli ji může vylepšit) je pochopitelné, že obsahuje všechny vlastnosti, které se od čtečky očekávají.
Čtečka nabízí několik typů zobrazení seznamů položek a poskytuje bohaté možnosti přizpůsobení: vzhledu, štítků, klávesových zkratek, filtrování.

Při prvním spuštění čtečky jsme zvolili možnost importu seznamu zdrojů; všechny zdroje (přibližně sto) se úspěšně vložily.
Okamžitě začala jejich kontrola, která vytvořila 2000 položek; při následné práce s nimi jsme nepozorovali žádné zpomalení.
Po instalaci nám však nefungovaly klávesové zkratky a nepodařilo se nám je ani zprovoznit.

Čtečka poskytuje snad každou vlastnost, na kterou uživatel pomyslí.
To je také důvod, proč ji nepovažujeme za vhodnou pro úplné začátečníky.
Začátečník, který nemá s tímto typem softwaru větší zkušenost, se může lehce ztratit v záplavě možností, byť čtečka obsahuje průvodce, který je všechny krátce představí.

\subsection{Opera}

\projekt{Opera}{http://www.opera.com/} je webový prohlížeč, který nabízí netradiční integraci několika dalších komponent.
Nás bude především zajímat Opera Mail, která umí sledovat RSS a \zkratka{Atom}{formát určený k popisu změn webových stránek; je novější alternativou k RSS~\cite{rfc4287}} kanály.
Možnosti této čtečky jsou značně omezené, umožňuje pouze nastavit interval kontroly každého zdroje a rozložení obrazovky; jednotlivé položky je možné třídit přidáním štítků.

Testování této čtečky pro nás bylo spíše zklamáním, chyběly nám některé z důležitých vlastností.
Na druhou stranu oceňujeme integraci do prohlížeče, které nemáme co vytknout: kanály je možné do čtečky přidávat kliknutím na ikonku v adresním řádku, celý článek se zobrazuje v novém listu.
Pokud je uživatel zvyklý na vzhled tohoto prohlížeče, bude spokojený i se vzhledem a chováním čtečky.

Tuto čtečku můžeme doporučit uživatelům, kteří prohlížeč Opera již používají; nemyslíme si, že samotná čtečka přiláká jiné uživatele.
Zároveň se domníváme, že čtečku budou využívat především příležitostní uživatelé, kteří na ni nebudou klást příliš velké nároky.
Jakmile začnou vyžadovat pokročilejší vlastnosti čtečky, budou se muset porozhlédnout po jiné.
Je to prostě jen příjemné rozšíření prohlížeče, které potěší běžné uživatele.

\subsection{Tiny Tiny RSS}

\projekt{Tiny Tiny RSS}{http://tt-rss.org/} je čtečka na pomezí webové služby a nativní aplikace.
Neexistuje žádný oficiální poskytovatel služby, u kterého by se uživatelé mohli registrovat, a není doporučené ji instalovat na osobní počítač.
Obvykle ji provozuje jednotlivec či skupina na vlastním serveru, často jím je domácí \zkratka{NAS}{Network-attached storage -- zařízení v lokální síti, které provozuje souborový server; obvykle je hardware a software takového zařízení uzpůsoben svému účelu}, či pronajatý (virtualizovaný) server.
Složitější způsob instalace slouží jako filtr potenciálních uživatelů.

Seznam položek je přehledný, lehce ovladatelný; všechny ovládací prvky se nacházejí na očekávaných místech.
Čtečka překypuje mnoha drobnými vylepšeními, které uživatele potěší a zjednoduší mu její používání.
Na druhou stranu, nefungovala nám taková základní funkcionalita, jakou je navigace v historii (tlačítka zpět a vpřed v prohlížeči).
Několikrát nám čtečka přestala reagovat, pomohlo až obnovení stránky; nevíme, zda šlo o chybu aplikace nebo problém s konfigurací.

Tiny Tiny RSS můžeme doporučit pokročilým technicky zdatným uživatelům, kteří mají rádi svá data pod kontrolou.
Pokud zvládnou složitější instalaci a konfiguraci, čtečka jim nabídne všechny vymoženosti, které poskytuje konkurence.

\subsection{Netvibes}

\projekt{Netvibes}{http://www.netvibes.com/} je webová služba, která je dostupná v základní verzi zdarma; pokročilejší verze jsou zpoplatněné.
Netvibes se snaží nabídnout uživatelům pohodlné a graficky vyvedené rozhraní; nabízí dva různé pohledy na sledované kanály.
První z nich má podobu nástěnky -- graficky bohatých bloků na stránce, které zobrazují několik posledních položek z vybraných kanálů.
Tento pohled je vhodný pro uživatele, kteří potřebují sledovat několik zdrojů najednou, paralelně vedle sebe.
Druhý z nabízených pohledů je tvořen klasickým seznamem všech položek, lze ho filtrovat podle zdroje, ze kterého položky pochází.
Tím však jeho možnosti nastavení končí.

Vyzkoušeli jsme do čtečky vložit seznam všech našich zdrojů, čtečka je během několika málo sekund načetla a začala poskytovat položky.
Potěšila nás dobrá podpora klávesových zkratek a přítomnost kontextové nápovědy zobrazující všechny aktuálně dostupné operace.
Zklamala nás ale téměř zcela chybějící podpora pro jakékoli ukládání a správu položek; čtečka umožňuje pouze označení položky k pozdějšímu přečtení.

Se čtečkou se nám nepracovalo příliš pohodlně, text přečtených položek je málo kontrastní, ikonky stavu aktuální položky jsou umístěné jinde než ikonky ostatních položek.
Grafické rozhraní nám občas přišlo pomalé a její dvojakost (nástěnka -- seznam) v nás zanechala silně rozpačité pocity.
Čtečku můžeme doporučit uživatelům, kterým se líbí vzhled aplikace a nevyžadují od ní pokročilé vlastnosti.

\subsection{Google Reader}

\projekt{Google Reader}{http://www.google.com/reader/about/} je čtečka vyvinutá společností Google v roce 2005; postupem času získávala podporu jednotlivých vlastností.
V březnu 2013 bylo oznámeno plánované ukončení provozu této čtečky; nyní již není dostupná~\cite{google-reader-down}.
Zmiňujeme ji především z důvodu, že inspirovala další tvůrce k vytvoření alternativ.

V době, kdy byla čtečka Google Reader ještě dostupná, se nám s ní pracovalo pohodlně; byla čtečkou, kterou jsme rutinně používali.
Měla svá omezení, především v práci se štítky; chyběly nám klávesové zkratky pro některé méně časté operace.
Musíme však ocenit její rychlost a přehledné rozhraní.
Její výhodou bylo také velké množství aplikací třetích stran, které z ní přebíraly položky a umožňovaly synchronizaci jejich stavu.

\subsection{The Old Reader}

\projekt{The Old Reader}{http://theoldreader.com/} je jednou z náhrad čtečky Google Reader; autoři ji začali vyvíjet v polovině roku 2012.
Důvodem vzniku byl jejich nesouhlas se směrem, kterým se čtečka od Googlu začala ubírat.
Po březnovém oznámení ukončení čtečky Google Reader začalo tuto čtečku používat mnohonásobně více lidí, to její další vývoj urychlilo.

Čtečku jsme vyzkoušeli a vložili do ní všechny naše zdroje.
Samotný import trval několik minut; již během tohoto importu byla čtečka použitelná; postupně přibývaly všechny zdroje.
Čtečka se nám z grafického hlediska líbí, její rozhraní je přehledné a ovládání logické.
Jediným nedostatkem, který jsme zaznamenali, je nedostatečné odlišení přečtených a nepřečtených položek (nepřečtené se liší pouze zeleným proužkem po straně).

Čtečka nabízí tři akce, které je možné vykonat s každou položkou: sdílet ji ve svém RSS kanálu, ponechat nepřečtenou a oblíbit si ji.
Pokud naši přátelé také používají tuto čtečku, je možné přímo sledovat položky, které označili ke sdílení.
Oblíbené položky slouží k ukládání položek, jiný mechanismus aplikace neposkytuje; uložené položky není možné nijak třídit.

Čtečku považujeme za povedenou; je vhodnou volbou pro uživatele hledající nástupce zrušené čtečky Google Reader.
Můžeme ji doporučit i uživatelům, kteří s čtečkami nemají velké zkušenosti; těm se může líbit možnost nalezení uživatelů, kteří mají v \uv{oblíbených} stejné položky.
Naopak, čtečka nejspíše nebude postačovat náročným uživatelům, kteří si chtějí katalogizovat zajímavé články a budovat si z nich vlastní knihovnu.

\subsection{Feedly}

\projekt{Feedly}{http://www.feedly.com/} je další z oblíbených náhrad ukončeného čtečky Google Reader; tato čtečka pravděpodobně zaznamenala příliv největší části jejích bývalých uživatelů.
Média tuto čtečku často vyzdvihují a zmiňují; obzvláště zřetelné to bylo v před-červencových článcích informujících o blížícím se ukončení čtečky poskytované společností Google.

Stejně jako ostatní, zkusili jsme i tuto čtečku nějakou dobu používat.
Rozhraní na nás jako celek působilo vlídně a přehledně; problém s používáním jsme zaznamenali u zobrazení položek.
Příliš velký prostor je vynechaný kolem samotného textu položky; ovládací prvky jsou nelogicky rozházené do několika skupin.
Velice nás obtěžovala přítomnost ikonek několika sociálních sítí, z nichž ani jednu nepoužíváme, nacházejí se totiž na místě, které by bylo vhodnější pro některé jiné, častější akce.
Čtečka umožňuje označit položku k pozdějšímu přečtení nevýraznou ikonkou, kterou je snadné přehlédnout.
Podporované je rovněž přiřazování štítků jednotlivým položkám, bohužel text štítku je špatně čitelný.

Feedly sice nabízí hodně možností, ale celkový dojem jí zhoršují detaily, které uživateli ztěžují běžnou práci.
Cílovou skupinou této čtečky jsou uživatelé, kteří hledají náhradu za Google Reader.
Čtečku můžeme doporučit jen těm, kteří nebyli spokojení s konkurenční čtečkou The Old Reader, od které se příliš neliší.

\section{Srovnání čteček}

V přehledné tabulce~\ref{tab:srovnani-ctecek} stručně shrneme výsledky našeho průzkumu čteček, srovnáváme je v několika kritériích.
Součástí tabulky je i námi návržená a naprogramovaná čtečka.

%~-
\begin{sidewaystable}
	\caption{Srovnání čteček}\label{tab:srovnani-ctecek}
	\tabulinesep=8pt 
	\begin{tabu} {| X[-1.5,l,p] || X[-1.5,l,p] | X[-2,l,p] | X[-2,l,p] | X[-1.2,l,p] |}
		\rowfont{\bfseries}
		\hline
		Čtečka & 
		Typ & 
		Cílový uživatel & 
		Ukládání položek &
		Klávesové zkratky \\ 
		\hline
		\hline
		\textbf{RSSOwl} & 
		nativní aplikace & 
		náročný uživatel vyžadující čtečku bez kompromisů & 
		přiřazení štítků, zvýraznění položky, přesun/kopírování položek do adresářů &
		kompletní, konfigurovatelné \\
		\hline
		\textbf{Opera} & 
		nativní aplikace -- internetový prohlížeč & 
		uživatel Opery, který nevyžaduje pokročilé vlastnosti & 
		přiřazení štítků, připnutí položky & 
		základní, konfigurovatelné \\
		\hline
		\textbf{Tiny Tiny RSS} & 
		webová služba, která vyžaduje vlastní server & 
		náročný uživatel, který chce mít data pod kontrolou & 
		přiřazení štítků, přidání hvězdičky &
		kompletní \\
		\hline
		\textbf{Netvibes} &
		webová služba &
		běžný uživatel, který potřebuje sledovat zároveň více zdrojů &
		přečíst později &
		kompletní \\
		\hline
		\textbf{Google Reader} &
		webová služba & 
		ukončena; všechny typy uživatelů & 
		přiřazení štítků, přidání hvězdičky &
		téměř kompletní \\
		\hline
		\textbf{The Old Reader} & 
		webová služba & 
		nepříliš náročný uživatel & 
		oblíbené položky &
		kompletní \\
		\hline
		\textbf{Feedly} & 
		webová služba & 
		uživatelé, kterým nevyhovuje The Old Reader & 
		přiřazení štítků, přečíst později &
		téměř kompletní \\
		\hline
		\hline
		\textbf{Naše aplikace} &
		webová služba &
		náročný uživatel &
		přiřazení štítků &
		pokročilé, konfigurovatelné \\
		\hline
	\end{tabu}
\end{sidewaystable}
%~+
