\chapter{Webové čtečky}

% TODO definice RSS
% TODO definice RSS kanál

Na internetu je dnes dostupné poměrně velké množství nejrůznějšího software a nejrůznějších služeb, které poskytují svým uživatelům přehledný výtah z článků, které se objeví na několika webových portálech.
Takové aplikace bývají často zaměřeny na různé cílové skupiny: některé se zaměřují na uživatele, kteří se chtějí o novém článku dozvědět okamžitě, jakmile je publikován, jiné zaujmou spíše uživatele, kteří mají příliš mnoho zdrojů na to, aby bylo v jejich silách je pravidelně navštěvovat; v neposlední řadě existují aplikace, které se snaží zvýšit komfort čtení článků svým uživatelům.
V této kapitole se pokusíme rozebrat charakteristiky jednotlivých aplikací.

\section{Typy čteček}

Před vlastním dělením aplikací vyčleníme ty, které obsahují čtečku RSS jen jako doplněk nebo ho využívají (jen) jako informační kanál o svých změnách.
% TODO odkaz na zdroj: XBMC RSS na úvodní obrazovce.
% TODO klikací odkaz na xbmc.org
Mezi takové aplikace můžeme zařadit například multimediální centrum XBMC, které obsahuje na úvodní obrazovce ve spodní části proužek s běžícím textem: tituly článků na domovské stránce xbmc.org.
Přestože XBMC má editor RSS kanálů a je možné nastavit vlastní kanál, nemůžeme aplikace takového typu považovat za plnohodnotné RSS čtečky.
Touto a podobnými aplikacemi se zabívat nebudeme; nebudou nás zajímat ani služby, které sice umožňují sledovat RSS, ale není to jejich primárním cílem.
% TODO linky na seznam a igoogle
Příkladem takové služby může být titulní stránka vyhledávače seznam.cz, kterou si uživatelé mohou upravit a přidat svůj RSS kanál, jehož položky se na stránce budou zobrazovat, další podobnou službou byl -- nyní ukončený -- iGoogle.

\bigskip

V následujících podkapitolách rozdělíme čtečky podle několika kritérií:
\begin{itemize}
	\item způsobu integrace do uživatelského prostředí,
	\item uživatele, na kterého jsou zaměřené.
\end{itemize}
Bude zřejmé, že výhody jednoho typu čteček jsou často nevýhodami jinýho typu.

\subsection{Nativní aplikace}

% TODO agregátor RSS kanálů
Pro používání toho typu čteček je nutné nainstalovat nativní aplikaci do operačního systému.
S tím souvisí několik omezení, se kterými musí uživatel počítat, například: aplikace musí být trvale spuštěná, neboť kontroluje zdroje jednotlivých kanálů.

%TODO rozvést nativní aplikace

Mezi nativní aplikace budeme řadit i samotný internetový prohlížeč, pokud umí sledovat RSS kanály.

\subsubsection{Výhody}

Mezi výhody nativní aplikace patří:
\begin{itemize}
    \item rychlejší odezva.
        Aplikace může ukládat všechny položky na pevný disk, čímž odbourává nutnost neustálé komunikace a opakovaného stahování veškerých dat ze serveru poskytovatele služby.
    \item offline čtení.
        Aplikace může stáhnout kompletní webovou stránku, na kterou se uživatel může podívat a přečíst originální text i v době, kdy nemá přístup k internetu.
		Toho mohou využívat třeba uživatelé, kteří často cestují a chtějí si zkrátit čtením dlouhé čekání.
    \item nezávislost na jiném subjektu.
        Uživatel není závislý na provozovateli služby; webové služby mohou skončit nebo mít výpadek.
        U nativní aplikace maximálně hrozí zastavení jejího dalšího vývoje, což zpravidla nemívá vliv na její dostupnost a funkcionalitu.
\end{itemize}

\subsection{Webová služba}

Přesným opakem nativních aplikací jsou webové služby; pro používání webové služby není vyžadována instalace aplikace, data jsou dostupná na kterémkoli zařízení s internetem.
Největší slabinou těchto čteček je nutnost neustálého připojení na internet.

\subsubsection{Výhody}

Mezi výhody aplikací postavených na využití webových služeb patří:
\begin{itemize}
    \item synchronizace stavu mezi více počítači.
        Uživatel má přístup ke svým sledavaným zdrojům a odkazům na články odkudkoli; může se přesouvat mezi několika zařízeními a pokračovat v procházení seznamu článků.
    \item menší riziko přetížení zdroje.
        % TODO diskuse o weather undergoundu na xbmc
        V případě, že jeden zdroj sleduje velké množství uživatelů, stačí jeden jediný dotaz pro stažení RSS dokumentu a uspokojení všech uživatelů najednou.
        S možností přetížení můžou mít problém nativní aplikace.
		Tento problém nastala například poskytovali počasí, z jehož serverů si aplikace XBMC stahovala informace příliš často (každých 30 minut).
    \item není nutné přepínat mezi aplikacemi.
        Pokud uživatel prochází seznam člán\-ků na stánce webové služby, nemusí se pro zobrazení originálního článku přepínat do internetového prohlížeče.
        Toto omezení neplatí pro samotné internetové prohlížeče a je redukované pro některé nativní aplikace, které umožňují zobrazit zjednodušenou webovou stránku přímo uvnitř sebe.
    \item uživatel má dostupné všechny položky bez ohledu na četnost spuštění počítače.
        Uživatel nemusí být stále online, webová služba kontroluje zdroj RSS bez ohledu na stav svých uživatelů.
        Pokud uživatel nativní aplikace nespustí čtečku po dobu několika dnů, může ztratit informace o některých článcích, zejména u frekventovaných zdrojů.
\end{itemize}

\subsection{Běžný uživatel}
% FIXME běžný uživatel

\subsection{Náročný uživatel}

Za náročného uživatele budeme považovat uživatele, který používá čtečku z důvodu, že již sám není schopný navštěvovat pravidelně stránky svých informačních zdrojů.
Důvodem může být sledování příliš mnoha zdrojů, na kterých je publikováno mnoho článků každý den.
\uv{Mnoho} v tomto kontextu znamená desítky zdrojů a stovky článků či zpráv denně.

Náročný uživatel klade na čtečku specifické nároky:
\begin{itemize}
    \item Vyžaduje kompaktní zobrazení, nesmí se plýtvat místem.
        Obrazovka musí být vyhrazená seznamu položek; každá položka musí zabírat minimální prostor, aby se jich vešlo na obrazovku co nejvíce.
    \item Preferuje prostý text před grafickými efekty; nic nesmí rozpylovat jeho pozornost.
    \item Vyžaduje aplikaci, která bude převážně ovladatelná klávesovými zkratkami; pohyb a klikání myši je pomalé.
        Náročný uživatel ve čtečce tráví spoustu času, který musí využít efektivně.
    \item Nepotřebuje zjednodušené prostředí, protože investice do podrobné konfigurace se mu časem vyplatí.
\end{itemize}

\section{Existující čtečky}

% TODO existující čtečky

\section{Srovnání čteček}

% TODO srovnání čteček
