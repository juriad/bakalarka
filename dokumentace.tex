\chapter{Vyskokoúrovňová dokumentace}

V této části práce popíšeme funkci/implementaci nejdůležitějších částí aplikace.
Tento popis je určen programátorům; má za cíl poskytnout dostatečný přehled o všech částech aplikace před tím, než se setká se zdrojovým kódem.

\section{Členění aplikace}

Jak jsme již popsali v předchozích kapitolách, aplikace se člení na tři hlavní části:
\begin{enumerate}
	\item serverovou část,
	\item klientskou část,
	\item nepovinný doplněk do prohlížeče.
\end{enumerate}

\begin{figure}
    \centering
    \includegraphics[width=12cm]{img/cleneni.eps}
    \caption{Vztahy mezi zdrojovým kódem a výsledným produktem kompilace}
    \label{fig:cleneni}
\end{figure}

První dvě části jsou naprogramovány v jazyku Java a sdílejí některé části kódu -- balík \verb|cz.artique.shared|.
Na obrázku \ref{fig:cleneni} jsou zobrazeny jednotlivé balíky, postup kompilace a její výsledný produkt.

Serverová část závisí na všech balících; můžeme však říct, že závislost na balíku \verb|cz.artique.client| je slabá a vynucená použitou frameworkem GWT.
GWT vyžaduje, aby rozhraní služby a jeho asynchronní varianta byly ve stejném balíku.
Samotná rozhraní jsou sice implementována jen v balíku \verb|cz.artique.server|, nicméně je nutné je mít přístupné v klientské části.

Sdílená část zdrojového kódu obsahuje především definici modelu -- tříd reprezentujících strukturu dat v databázi podle návrhu entit.
Jednotlivým entitám odpovídá jedna nebo více tříd.

Kromě modelu obsahuje sdílená část rozhraní implementovaná třídami modelu a pomocné třídy používané při komunikaci mezi serverovou a klientskou částí.

\bigskip{}

Doplněk do prohlížeče je vytvořen zcela odděleně od zbytku aplikace v jazyku JavaScript.

\section{Rozhraní serverové části}

Serverová část je zodpovědná za abstrakci přístupu k databázi, poskytování služeb klientům a vykonávání periodických procesů.

\subsection{Obsluha databáze}
Jelikož platformou poskytovaná databáze neumožňuje složitější operace, navrhli jsme kolekci služeb, které obalují přístup k databázi a provádějí jednoduché manipulace.
Přístup k databázi ze zbytku splikace směřujeme přes tyto služby.
To nám umožňuje mít přehled o typech prováděných operací a zároveň eliminuje skryté závislosti.
Nejčastějšími operacemi, které takto simulujeme, jsou:
\begin{itemize}
	\item dotažení dalšího objektu podle klíče, například: k uživatelskému zdroji dotáhne region,
	\item vytáhnutí objektu, příp. jeho vytvoření, pokud neexistuje,
	\item sestavení složitějšího dotazu podle kritérii.
\end{itemize}

\subsection{Rozhraní pro komunikaci s klientskou částí}

Klientská část komunikuje po síti internet se serverovou částí pomocí několika služeb.
Tyto služby lze rozpoznat podle jejich názvu začínajícího slovem \verb|Client|.

\subsubsection{Validace vstupu}

Všechny služby, které jsou veřejně dostupné -- mezi ně patří i služby volané z klientské části -- vždy, bez výjimky, provádějí ošetření hodnot vstupů.
Některé typy validace jsou prováděné z důvodu bezpečnosti, jiné z důvodu zajištění integrity dat či omezení danými použitými knihovnami a databází.
Prováděné úpravy a kontroly hodnot jsou následující:
\begin{itemize}
	\item nahrazení hodnotami dopočítatelnými na straně serveru hodnoty, které dodala klientská část.
		Sem patří například:
		\begin{itemize}
			\item výpočet databázového klíče,
			\item dosazení typu z výčtu, pokud existuje jediná možnost,
			\item dosazení aktuálního datumu do datumových atributů,
			\item nahrazení vlastníka za aktuálně přihlášeného uživatele.
		\end{itemize}
	\item kontrola, zda povinný atribut je vyplněný,
	\item kontrola, zda vlastník všech odkázaných objektů je aktuálně přihlášený uživatel,
	\item oříznutí bílých znaků z řetězců a následná kontrola délky (databáze povoluje řetězce pouze o maximální délce 500 znaků),
	\item oříznutí bílých znaků z delších textů,
	\item kontrola, zda URL je validní,
	\item kontrola, zda URL je validní a webová stránka jí určená je dostupná,
	\item kontrola, zda CSS selector je validní,
	\item oříznutí bílých znaků z názvu štítku a kontrola, zda název štítku neobsahuje bílé znaky a znak \$.
\end{itemize}

Pokud nebudeme uvažovat validaci, která samotná často zabírá několik desítek řádků kódu, zjistíme, že mnoho z dále popsaných služeb je trivialních.
Tyto služby většinou spočívají jen v zavolání metody poskytnuté obsluhou databáze a vrácení jejího výsledku.

\bigskip{}
Dále popíšeme rozhraní všech služeb.

\subsubsection{Přihlášení, odhlášení}

Služba \verb|ClientUserService| poskytuje klientské části informace o aktuálně přihlášeném uživateli skrze jedinou metodu -- \verb|login|.
Pokud je uživatel přihlášen, vrací metoda objekt uživatele a internetovou adresu určenou k jeho odhlášení.
Součástí objektu uživatele je i klientský token, který se používá při komunikaci s klientským doplňkem. % TODO klientský doplněk
Zároveň zkontroluje, jestli existuje manuální zdroj pro daného uživatele: pokud neexistuje (uživatel se přihlásil poprvé), vytvoří se.
Naopak, pokud není uživatel přihlášen, vrací jen adresu určenou k přihlášení.

Informaci o aktuálně přihlášeném uživateli přebíráme ze služby \verb|UserService|, kterou nám poskytuje platforma Google AppEngine.

\subsubsection{Kontrola spojení}

Služba \verb|ClientPingService| nemá při běžném chodu aplikace žádný význam, ten získá až v okamžiku, kdy dojde k chybě.
Jakákoli služba pro komunikaci mezi klientskou a serverovou částí může selhat z nejrůznějších důvodů; jedním z důvodů je vypršení času určeného k vyčkání na odpověď.
Pokud dojde k výjimce \verb|RequestTimeoutException|, zkusí se, zda je serverová část vůbec dostupná pomocí metody \verb|ping| této služby.
Její implementace je záměrně triviální z důvodu vyloučení jiných možných chyb.

\subsubsection{Konfigurace}

Služba \verb|ClientConfigService| poskytuje klientské části metody pro získání seznamu všech konfiguračních položek pro aktuálního uživatele.
V případě, že konfigurační položka v databázi neexistuje, vrátí se klientovi s výchozí hodnotou.

Dále tato služba umožňuje dávkovou aktualizaci hodnot několik konfiguračních položek.
Novou hodnotu konfigurační položky uloží služba do databáze pro aktuálně přihlášeného uživatele.

Tato služba v současné době není plně využívána; navrhli jsme ji obecněji, abych si zajistili možnost pozdějšího rozšíření aplikace.

\subsubsection{Zdroje}

Veškeré manipulace jak se zdroji, tak i s uživatelskými zdroji probíhají skrze službu \verb|ClientSourceService|.
Zvolili jsme integraci manipulace s několika entitami do jedné jediné služby z důvodu, že klientská část nepracuje se zdroji téměř nikdy přímo, ale výhradně prostřednictvím uživatelských zdrojů.
Tato služba poskytuje následující metody:
\begin{itemize}
	\item přidat zdroj -- vytvoří v databázi zdroj daného typu, pokud ještě neexistoval a vrátí jej.
		Tato metoda se volá v situaci, kdy uživatel vytváří svůj nový uživatelský zdroj: při vytváření uživatelského zdroje je nutná znalost zdroje, který je personalizován.
	\item přidat uživatelský zdroj -- vytvoří v databázi uživatelský zdroj, který je personalizací existujícího zdroje.
		Tato metoda se volá při vytváření uživatelského zdroje po metodě přidat zdroj.
	\item aktualizovat uživatelský zdroj -- aktualizuje v databázi existující uživatelský zdroj.
	\item získat regiony -- vrátí seznam existujících regionů ke zdroji; zdroj musí být typu, který podporuje regiony.
		Tato metoda se volá v okamžiku načtení detailu uživatelského zdroje: buď po jeho vytvoření, nebo během jeho úpravy.
	\item zkontrolovat region -- zkontroluje, zda region je platný.
		Tato metoda může být volitelně zavolána před aktualizací uživatelského zdroje; stejná kontrola se provádí během samotné aktualizace.
	\item naplánovat kontrolu zdroje -- naplánuje okamžitou kontrolu zdroje.
		Mezi naplánováním a spuštěním kontroly může uběhnout časový interval definovaný frekvencí periodické kontroly zdrojů.
		Tato metoda je určena uživatelům pro případ, že se dozví jiným způsobem, že existují nové položky na zdroji dříve, než proběhne jeho řádná naplánovaná kontrola.
	\item získat doporučení -- vrátí předpočítaný seznam doporučených zdrojů pro aktuálního uživatele.
		V situaci, kdy uživatel vytváří nový zdroj, si může vybrat, zda přidá zdroj podle URL adresy či si vybere některý z jemu doporučených.
\end{itemize}

\subsubsection{Štítky}

Veškeré manipulace se štítky probíhají skrze službu \verb|ClientLabelService|.
Jelikož jsme zavedli silná omezení na štítky, je tato služba spíše jednodušší:
\begin{itemize}
	\item získat všechny štítky -- vrátí všechny štítky z databáze, které patří aktuálně přihlášenému uživateli, bez ohledu na jejich typ.
	\item přidat štítek -- vytvoří nový štítek v databázi, pokud štítek se stejným názvem ještě neexistuje a vrátí jej.
		Tuto metodu volá klientská část v situacích, kdy uživatel má možnost vybrat štítek a uživatel zadá název neexistujícího štítku.
	\item aktualizovat štítky -- provede dávkovou aktualizaci v databázi několika štítků; nastavením atributu \verb|toBeDeleted| lze štítek odstranit.
		Odstranění štítku je podmíněné tím, že není nikde ve zbytku aplikace použitý; v opačném případě odstranění selže.
\end{itemize}

\subsubsection{Seznamové filtry}

Služba \verb|ClientListFilterService| provádí veškeré manipulace se seznamovými filtry.
Nabízí obvyklé metody:
\begin{itemize}
	\item získat všechny seznamové filtry -- vrátí z databáze seznam všech seznamových filtrů, které patří aktuálně přihlášenému uživateli.
	\item přidat seznamový filtr -- uloží seznamový filtr do databáze.
	\item aktualizovat seznamový filtr -- aktualizuje v databázi seznamový filtr.
	\item odstranit seznamový filtr -- odstraní z databáze seznamový filtr.
\end{itemize}

\subsubsection{Položky}

Nejkomplexnější a nejdůležitější službou, kterou serverová část poskytuje klientské části je \verb|ClientItemsService|.
Tato služba poskytuje 3 metody, které manipulují s položkami:
\begin{itemize}
	\item Přidat manuální položku -- vytvoří v databázi novou manuální položku a odpovídající uživatelskou položku k manuálnímu zdroji aktuálně přihlášeného uživatele.
	\item Získat položky -- vrátí seznam uživatelských položek, které vyhovují seznamovému filtru.
		Pokud nejde o první dotaz, bude v dotazu uvedený i klíč poslední uživatelské položky, kterou klientská část zná; vráceny budou jen novější/starší uživatelské položky (závisí na způsobu řazení).
		V případě, že seznam má být řazený sestupně, bude výsledek obsahovat i seznam uživatelských položek, které jsou novější než klientské části známá nejnovější uživatelská položka.
		Výsledek obsahuje i informaci o tom, zda mohou existovat ještě další uživatelské položky -- jestli existuje šance, že následující dotaz vrátí neprázdný výsledek.
	\item Aktualizovat položky -- aktualizuje v databázi uživatelské položky na základě souboru změn, které se na ně mají aplikovat.
		Soubor změn udává, které štítky se mají uživatelské položce přidat či odebrat a jaký má být nový stav přečtenosti.
		Metoda vrací seznam uživatelských položek, které byly změněny.
\end{itemize}

\subsubsection{Klávesové zkratky}

Klávesová zkratka je jednoduchá entita, se kterou nepožadujeme provádění složitějších manipulací; vystačíme si s možností vytvoření a odstranění klávesové zkratky.
Tato služba poskytuje následující metody:
\begin{itemize}
	\item získat všechny klávesové zkratky -- vrátí z databáze seznam všech klávesových zkratek aktuálně přihlášeného uživatele.
	\item vytvořit klávesovou zkratku -- vytvoří v databázi klávesovou zkratku.
	\item odstranit klávesovou zkratku -- odstraní klávesovou zkratku z databáze.
\end{itemize}

Možnost změny klávesové zkratky se ukázala příliš náročnou na implementaci, proto jsme ji nahradili za posloupnost odstranění staré zkratky a následného vytvoření nové zkratky.

\subsection{Rozhraní pro komunikaci s klientským doplňkem}

Služba pro komunikaci s klientským doplňkem -- \verb|ClientServlet| vyžaduje přihlášení uživatele.
Užití mechanismu ověření proti Google Account se nám nepodařilo jednoduše implementovat.
Z toho důvodu jsme se rozhodli přidělit každému uživateli náhodný jednoznačný identifikátor, který bude uživatelům doplněk v prohlížeči používat při komunikovat se serverovou částí aplikace ke své identifikaci.
Tento identifikátor -- token -- je určen jen a pouze pro komunikaci s doplňkem.
Myslíme si, že jde o rozumný kompromis mezi zajištěním bezpečí a jednoduchostí implementace.

Rozhraní pro komunikaci s doplňkem nabízí dvě metody:
\begin{itemize}
	\item získat štítky -- vrátí z databáze seznam názvů všech uživatelských štítků.
		Funguje velice podobně jako metoda získat všechny štítky služby štítků pro komunikaci s klientskou částí.
		Liší se jen tím, že tato metoda filtruje štítky podle typu a vrací ještě seznam výchozích štítků pro manuální zdroj.
%TODO definovat JSON
	\item přidat položku -- přidá položku popsanou JSON objektem k manuálním zdroji uživatele.
		Provádí podobné kontroly a úpravy jako metoda přidat manuální položky služby položek pro komunikaci s klientskou částí.
		Jelikož toto rozhraní je textové (nepřenáší se klíče štítků, ale jejich názvy), může uživatel přidělit položce neexistující štítek.
		V takovém případě bude v okamžiku, kdy je zřejmé, že objekt položky prošel všemi kontrolami, vytvořen a následně k uživatelské položce přiřazen.
\end{itemize}

Pro použití tohoto rozhraní musí klient poskytnout svůj klientský token a uvést akci (metodu), která se má provést.

\subsection{Rozhraní pro plánování a plánované úlohy}

Do této skupiny jsme zařadili služby, které jsou spouštěny:
\begin{itemize}
		%TODO zadefinovat cron
	\item buď periodicky cronem (kontrola zdrojů, výpočet doporučení),
	\item nebo jsou realizovány pomocí úloh, které se postupně zpracovávají (kontrola zdrojů, zálohování položky)
\end{itemize}

\subsubsection{Kontrola zdrojů}

Naše aplikace vyžaduje pro získávání nového obsahu periodické spouštění kontroly zdrojů.
Cronem spouštěná služba nejprve zjistí seznam zdrojů, které je třeba zkontrolovat: to jsou takové, jejichž plánovaný čas kontroly je menší než aktuální čas, nenastalo u nich příliš mnoho chyb během poslední kontroly a zároveň nejsou právě kontrolovány.
Pro každý nalezený zdroj se vytvoří úloha, která je následně vložena do fronty.
Fronta úloh se stará o to, aby byly jednotlivé úlohy spouštěny postupně s určitou frekvencí, čímž se snaží snížit okamžitou zátěž.
Dalšími výhodami naplánování úloh oproti okamžitému spuštění kontroly zdroje jsou:
\begin{itemize}
	\item možnost kontroly několika zdrojů zároveň -- každá kontrola probíhá ve vlastním vlákně,
	\item maximální doba běhu skriptu se týká každé kontroly zvlášť, nehrozí tak vypršení časového limitu.
	\item možnost opakování kontroly zdroje v případě, že dojde k chybě.
\end{itemize}

Aby se aplikace lépe vypořádala s chybnými zdroji (nekontrolovala je nepříčetně často), po několika chybách při kontrole zastavíme jeho další kontroly v normálním režimu.
Takový zdroj bude kontrolován v chybovém režimu, méně často, jen jednou za 12 hodin; v případě, že se následná kontrola zdaří, vrátí se zdroj do normálního režimu.

\subsubsection{Výpočet doporučení}

% TODO výpočet doporučování
Popíšeme v samostatné podkapitole.

\subsubsection{Zálohování položky}

Kdykoli je nějaké uživatelské položce přiřazen štítek, zkontroluje se, zda štítek má nastavenou vlatnost zálohování.
Pokud ji má, vytvoří se nová zálohovací úloha, která se zařadí do fronty.
Stejně jako v případě kontroly zdrojů, i zde se fronta úloh stará o jejich postupné spouštění.
Postup zálohování jsme popsali v procesech. % TODO ref

\subsection{Veřejné rozhraní k obsahu}

% TODO veřejné rozhraní

\subsubsection{Exportování položek}

\subsubsection{Zálohované položky}

\section{Výpočet doporučování}

% TODO podrobně popsat výpočet doporučování
