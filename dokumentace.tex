\chapter{Dokumentace}

V této části práce popíšeme funkci/implementaci nejdůležitějších částí aplikace.
Tento popis je určen programátorům; má za cíl poskytnout dostatečný přehled o všech částech aplikace před tím, než se setká se zdrojovým kódem.

\section{Členění aplikace}

Jak jsme již popsali v předchozích kapitolách, aplikace se člení na tři hlavní části:
\begin{enumerate}
	\item serverovou část,
	\item klientskou část,
	\item nepovinný doplněk do prohlížeče.
\end{enumerate}

\begin{figure}
    \centering
    \includegraphics[width=12cm]{cleneni.eps}
    \caption{Vztahy mezi zdrojovým kódem a výsledným produktem kompilace}
    \label{fig:cleneni}
\end{figure}

První dvě části jsou naprogramovány v jazyku Java a sdílejí některé části kódu -- balík \verb|cz.artique.shared|.
Na obrázku \ref{fig:cleneni} jsou zobrazeny jednotlivé balíky, postup kompilace a její výsledný produkt.

Serverová část závisí na všech balících; můžeme však říct, že závislost na balíku \verb|cz.artique.client| je slabá a vynucená použitou frameworkem GWT.
GWT vyžaduje, aby rozhraní služby a jeho asynchronní varianta byly ve stejném balíku.
Samotná rozhraní jsou sice implementována jen v balíku \verb|cz.artique.server|, nicméně je nutné je mít přístupné v klientské části.

Sdílená část zdrojového kódu obsahuje především definici modelu -- tříd reprezentujících strukturu dat v databázi podle návrhu entit.
Jednotlivým entitám odpovídá jedna nebo více tříd.

Kromě modelu obsahuje sdílená část rozhraní implementovaná třídami modelu a pomocné třídy používané při komunikaci mezi serverovou a klientskou částí.

\bigskip

Doplněk do prohlížeče je vytvořen zcela odděleně od zbytku aplikace v jazyku JavaScript.

\section{Rozhraní serverové části}

Serverová část je zodpovědná za abstrakci přístupu k databázi, poskytování služeb klientům a vykonávání periodických procesů.

\subsection{Obsluha databáze}
Jelikož platformou poskytovaná databáze neumožňuje složitější operace, navrhli jsme kolekci služeb, které obalují přístup k databázi a provádějí jednoduché manipulace.
Přístup k databázi ze zbytku splikace směřujeme přes tyto služby.
To nám umožňuje mít přehled o typech prováděných operací a zároveň eliminuje skryté závislosti.
Nejčastějšími operacemi, které takto simulujeme, jsou:
\begin{itemize}
	\item dotažení dalšího objektu podle klíče, například: k uživatelskému zdroji dotáhne region,
	\item vytáhnutí objektu, příp. jeho vytvoření, pokud neexistuje,
	\item sestavení složitějšího dotazu podle kritérii.
\end{itemize}

\subsection{Rozhraní pro komunikaci s klientskou částí}

Klientská část komunikuje po síti internet se serverovou částí pomocí několika služeb.
Tyto služby lze rozpoznat podle jejich názvu začínajícího slovem \verb|Client|.

\subsubsection{Validace vstupu}

Všechny služby, které jsou veřejně dostupné -- mezi ně patří i služby volané z klientské části -- vždy, bez výjimky, provádějí ošetření hodnot vstupů.
Některé typy validace jsou prováděné z důvodu bezpečnosti, jiné z důvodu zajištění integrity dat či omezení danými použitými knihovnami a databází.
Prováděné úpravy a kontroly hodnot jsou následující:
\begin{itemize}
	\item nahrazení hodnotami dopočítatelnými na straně serveru hodnoty, které dodala klientská část.
		Sem patří například:
		\begin{itemize}
			\item výpočet databázového klíče,
			\item dosazení typu z výčtu, pokud existuje jediná možnost,
			\item dosazení aktuálního datumu do datumových atributů,
			\item nahrazení vlastníka za aktuálně přihlášeného uživatele.
		\end{itemize}
	\item kontrola, zda povinný atribut je vyplněný,
	\item kontrola, zda vlastník všech odkázaných objektů je aktuálně přihlášený uživatel,
	\item oříznutí bílých znaků z řetězců a následná kontrola délky (databáze povoluje řetězce pouze o maximální délce 500 znaků),
	\item oříznutí bílých znaků z delších textů,
	\item kontrola, zda URL je validní,
	\item kontrola, zda URL je validní a webová stránka jí určená je dostupná,
	\item kontrola, zda CSS selector je validní,
	\item oříznutí bílých znaků z názvu štítku a kontrola, zda název štítku neobsahuje bílé znaky a znak \$.
\end{itemize}

Pokud nebudeme uvažovat validaci, která samotná často zabírá několik desítek řádků kódu, zjistíme, že mnoho z dále popsaných služeb je trivialních.
Tyto služby většinou spočívají jen v zavolání metody poskytnuté obsluhou databáze a vrácení jejího výsledku.

\bigskip
Dále popíšeme rozhraní všech služeb.

\subsubsection{Přihlášení, odhlášení}

Služba \verb|ClientUserService| poskytuje klientské části informace o aktuálně přihlášeném uživateli skrze jedinou metodu -- \verb|login|.
Pokud je uživatel přihlášen, vrací metoda objekt uživatele a internetovou adresu určenou k jeho odhlášení.
Součástí objektu uživatele je i klientský token, který se používá při komunikaci s klientským doplňkem. % TODO klientský doplněk
Zároveň zkontroluje, jestli existuje manuální zdroj pro daného uživatele: pokud neexistuje (uživatel se přihlásil poprvé), vytvoří se.
Naopak, pokud není uživatel přihlášen, vrací jen adresu určenou k přihlášení.

Informaci o aktuálně přihlášeném uživateli přebíráme ze služby \verb|UserService|, kterou nám poskytuje platforma Google AppEngine.

\subsubsection{Kontrola spojení}

Služba \verb|ClientPingService| nemá při běžném chodu aplikace žádný význam, ten získá až v okamžiku, kdy dojde k chybě.
Jakákoli služba pro komunikaci mezi klientskou a serverovou částí může selhat z nejrůznějších důvodů; jedním z důvodů je vypršení času určeného k vyčkání na odpověď.
Pokud dojde k výjimce \verb|RequestTimeoutException|, zkusí se, zda je serverová část vůbec dostupná pomocí metody \verb|ping| této služby.
Její implementace je záměrně triviální z důvodu vyloučení jiných možných chyb.

\subsubsection{Konfigurace}

Služba \verb|ClientConfigService| poskytuje klientské části metody pro získání seznamu všech konfiguračních položek pro aktuálního uživatele.
V případě, že konfigurační položka v databázi neexistuje, vrátí se klientovi s výchozí hodnotou.

Dále tato služba umožňuje dávkovou aktualizaci hodnot několik konfiguračních položek.
Novou hodnotu konfigurační položky uloží služba do databáze pro aktuálně přihlášeného uživatele.

Tato služba v současné době není plně využívána; navrhli jsme ji obecněji, abych si zajistili možnost pozdějšího rozšíření aplikace.

\subsubsection{Zdroje}

Veškeré manipulace jak se zdroji, tak i s uživatelskými zdroji probíhají skrze službu \verb|ClientSourceService|.
Zvolili jsme integraci manipulace s několika entitami do jedné jediné služby z důvodu, že klientská část nepracuje se zdroji téměř nikdy přímo, ale výhradně prostřednictvím uživatelských zdrojů.
Tato služba poskytuje následující metody:
\begin{itemize}
	\item přidat zdroj -- vytvoří v databázi zdroj daného typu, pokud ještě neexistoval a vrátí jej.
		Tato metoda se volá v situaci, kdy uživatel vytváří svůj nový uživatelský zdroj: při vytváření uživatelského zdroje je nutná znalost zdroje, který je personalizován.
	\item přidat uživatelský zdroj -- vytvoří v databázi uživatelský zdroj, který je personalizací existujícího zdroje.
		Tato metoda se volá při vytváření uživatelského zdroje po metodě přidat zdroj.
	\item aktualizovat uživatelský zdroj -- aktualizuje v databázi existující uživatelský zdroj.
	\item získat regiony -- vrátí seznam existujících regionů ke zdroji; zdroj musí být typu, který podporuje regiony.
		Tato metoda se volá v okamžiku načtení detailu uživatelského zdroje: buď po jeho vytvoření, nebo během jeho úpravy.
	\item zkontrolovat region -- zkontroluje, zda region je platný.
		Tato metoda může být volitelně zavolána před aktualizací uživatelského zdroje; stejná kontrola se provádí během samotné aktualizace.
	\item naplánovat kontrolu zdroje -- naplánuje okamžitou kontrolu zdroje.
		Mezi naplánováním a spuštěním kontroly může uběhnout časový interval definovaný frekvencí periodické kontroly zdrojů.
		Tato metoda je určena uživatelům pro případ, že se dozví jiným způsobem, že existují nové položky na zdroji dříve, než proběhne jeho řádná naplánovaná kontrola.
	\item získat doporučení -- vrátí předpočítaný seznam doporučených zdrojů pro aktuálního uživatele.
		V situaci, kdy uživatel vytváří nový zdroj, si může vybrat, zda přidá zdroj podle URL adresy či si vybere některý z jemu doporučených.
\end{itemize}

\subsubsection{Štítky}

Veškeré manipulace se štítky probíhají skrze službu \verb|ClientLabelService|.
Jelikož jsme zavedli silná omezení na štítky, je tato služba spíše jednodušší:
\begin{itemize}
	\item získat všechny štítky -- vrátí všechny štítky z databáze, které patří aktuálně přihlášenému uživateli, bez ohledu na jejich typ.
	\item přidat štítek -- vytvoří nový štítek v databázi, pokud štítek se stejným názvem ještě neexistuje a vrátí jej.
		Tuto metodu volá klientská část v situacích, kdy uživatel má možnost vybrat štítek a uživatel zadá název neexistujícího štítku.
	\item aktualizovat štítky -- provede dávkovou aktualizaci v databázi několika štítků; nastavením atributu \verb|toBeDeleted| lze štítek odstranit.
		Odstranění štítku je podmíněné tím, že není nikde ve zbytku aplikace použitý; v opačném případě odstranění selže.
\end{itemize}

\subsubsection{Seznamové filtry}

Služba \verb|ClientListFilterService| provádí veškeré manipulace se seznamovými filtry.
Nabízí obvyklé metody:
\begin{itemize}
	\item získat všechny seznamové filtry -- vrátí z databáze seznam všech seznamových filtrů, které patří aktuálně přihlášenému uživateli.
	\item přidat seznamový filtr -- uloží seznamový filtr do databáze.
	\item aktualizovat seznamový filtr -- aktualizuje v databázi seznamový filtr.
	\item odstranit seznamový filtr -- odstraní z databáze seznamový filtr.
\end{itemize}

\subsubsection{Položky}

Nejkomplexnější a nejdůležitější službou, kterou serverová část poskytuje klientské části je \verb|ClientItemsService|.
Tato služba poskytuje 3 metody, které manipulují s položkami:
\begin{itemize}
	\item Přidat manuální položku -- vytvoří v databázi novou manuální položku a odpovídající uživatelskou položku k manuálnímu zdroji aktuálně přihlášeného uživatele.
	\item Získat položky -- vrátí seznam uživatelských položek, které vyhovují seznamovému filtru.
		Pokud nejde o první dotaz, bude v dotazu uvedený i klíč poslední uživatelské položky, kterou klientská část zná; vráceny budou jen novější/starší uživatelské položky (závisí na způsobu řazení).
		V případě, že seznam má být řazený sestupně, bude výsledek obsahovat i seznam uživatelských položek, které jsou novější než klientské části známá nejnovější uživatelská položka.
		Výsledek obsahuje i informaci o tom, zda mohou existovat ještě další uživatelské položky -- jestli existuje šance, že následující dotaz vrátí neprázdný výsledek.
	\item Aktualizovat položky -- aktualizuje v databázi uživatelské položky na základě souboru změn, které se na ně mají aplikovat.
		Soubor změn udává, které štítky se mají uživatelské položce přidat či odebrat a jaký má být nový stav přečtenosti.
		Metoda vrací seznam uživatelských položek, které byly změněny.
\end{itemize}

\subsubsection{Klávesové zkratky}

Klávesová zkratka je jednoduchá entita, se kterou nepožadujeme provádění složitějších manipulací; vystačíme si s možností vytvoření a odstranění klávesové zkratky.
Tato služba poskytuje následující metody:
\begin{itemize}
	\item získat všechny klávesové zkratky -- vrátí z databáze seznam všech klávesových zkratek aktuálně přihlášeného uživatele.
	\item vytvořit klávesovou zkratku -- vytvoří v databázi klávesovou zkratku.
	\item odstranit klávesovou zkratku -- odstraní klávesovou zkratku z databáze.
\end{itemize}

Možnost změny klávesové zkratky se ukázala příliš náročnou na implementaci, proto jsme ji nahradili za posloupnost odstranění staré zkratky a následného vytvoření nové zkratky.

\subsection{Rozhraní pro komunikaci s klientským doplňkem}

Služba pro komunikaci s klientským doplňkem -- \verb|ClientServlet| vyžaduje přihlášení uživatele.
Užití mechanismu ověření proti Google Account se nám nepodařilo jednoduše implementovat.
Z toho důvodu jsme se rozhodli přidělit každému uživateli náhodný jednoznačný identifikátor, který bude uživatelům doplněk v prohlížeči používat při komunikovat se serverovou částí aplikace ke své identifikaci.
Tento identifikátor -- token -- je určen jen a pouze pro komunikaci s doplňkem.
Myslíme si, že jde o rozumný kompromis mezi zajištěním bezpečí a jednoduchostí implementace.

Rozhraní pro komunikaci s doplňkem nabízí dvě metody:
\begin{itemize}
	\item získat štítky -- vrátí z databáze seznam názvů všech uživatelských štítků.
		Funguje velice podobně jako metoda získat všechny štítky služby štítků pro komunikaci s klientskou částí.
		Liší se jen tím, že tato metoda filtruje štítky podle typu a vrací ještě seznam výchozích štítků pro manuální zdroj.
%TODO definovat JSON
	\item přidat položku -- přidá položku popsanou JSON objektem k manuálním zdroji uživatele.
		Provádí podobné kontroly a úpravy jako metoda přidat manuální položky služby položek pro komunikaci s klientskou částí.
		Jelikož toto rozhraní je textové (nepřenáší se klíče štítků, ale jejich názvy), může uživatel přidělit položce neexistující štítek.
		V takovém případě bude v okamžiku, kdy je zřejmé, že objekt položky prošel všemi kontrolami, vytvořen a následně k uživatelské položce přiřazen.
\end{itemize}

Pro použití tohoto rozhraní musí klient poskytnout svůj klientský token a uvést akci (metodu), která se má provést.

\subsection{Rozhraní pro plánování a plánované úlohy}

Do této skupiny jsme zařadili služby, které jsou spouštěny:
\begin{itemize}
		%TODO zadefinovat cron
	\item buď periodicky cronem (kontrola zdrojů, výpočet doporučení),
	\item nebo jsou realizovány pomocí úloh, které se postupně zpracovávají (kontrola zdrojů, zálohování položky)
\end{itemize}

\subsubsection{Kontrola zdrojů}

Naše aplikace vyžaduje pro získávání nového obsahu periodické spouštění kontroly zdrojů.
Cronem spouštěná služba nejprve zjistí seznam zdrojů, které je třeba zkontrolovat: to jsou takové, jejichž plánovaný čas kontroly je menší než aktuální čas, nenastalo u nich příliš mnoho chyb během poslední kontroly a zároveň nejsou právě kontrolovány.
Pro každý nalezený zdroj se vytvoří úloha, která je následně vložena do fronty.
Fronta úloh se stará o to, aby byly jednotlivé úlohy spouštěny postupně s určitou frekvencí, čímž se snaží snížit okamžitou zátěž.
Dalšími výhodami naplánování úloh oproti okamžitému spuštění kontroly zdroje jsou:
\begin{itemize}
	\item možnost kontroly několika zdrojů zároveň -- každá kontrola probíhá ve vlastním vlákně,
	\item maximální doba běhu skriptu se týká každé kontroly zvlášť, nehrozí tak vypršení časového limitu.
	\item možnost opakování kontroly zdroje v případě, že dojde k chybě.
\end{itemize}

Aby se aplikace lépe vypořádala s chybnými zdroji (nekontrolovala je nepříčetně často), po několika chybách při kontrole zastavíme jeho další kontroly v normálním režimu.
Takový zdroj bude kontrolován v chybovém režimu, méně často, jen jednou za 12 hodin; v případě, že se následná kontrola zdaří, vrátí se zdroj do normálního režimu.

\subsubsection{Výpočet doporučení}

% TODO link na důvody přepočítávání doporučení offline
Seznam zdrojů doporučovaných jednotlivým uživatelům přepočítáváme z důvodů, které jsme zmínili v kapitole XXX, jen jednou za den.
O pravidelné spouštění procesu se stará cron poskytovaný platformou Google AppEngine.

Výpočet probíhá v několika krocích:
\begin{enumerate}
	\item Načtení všech uživatelských zdrojů z databáze a vytvoření matice uživatelů a zdrojů.
		Matice má v řádcích uživatele, ve sloupcích zdroje.
		Prvek matice vyplníme:
		\begin{itemize}
			\item buď jedničkou, pokud existuje uživatelský zdroj – uživatel sleduje zdroj,
			\item nebo nulou v opačném případě.
		\end{itemize}
		Tvorbou matice z uživatelských zdrojů si zajistíme přítomnost alespoň jedné jedničky v každém řádku i sloupci.
		Matici nazveme $M$.
	\item Matici normalizujeme zvlášť v řádcích a sloupcích: každý řádek, resp. sloupec vydělíme jeho součtem.
		Tato normalizace nám zajistí rovnost vah všech uživatelů, resp. zdrojů (jinak by měl větší váhu uživatel s mnoha zdroji resp. zdroj sledovaný mnoha uživateli).
		Matici s normalizovanými zdroji pro uživatele nazveme $M_u$.
		Matici s normalizovanými uživateli pro zdroje nazveme $M_z$
	\item Připravíme si čtvercovou matici popisující vhodnost uživatelů: jak jsou si uživatelé blízcí.
		Před začátkem výpočtu je tato matice jednotková; zatím neznáme blízké uživatele (reflexivita zajišťuje stabilitu výpočtu).
		Matici označíme $U_0$; $U = U_0$ bude matice používaná během výpočtu.
	\item Provedeme krok výpočtu:
		aktualizujeme nejprve váhy zdrojům pro všechny uživatele:
			$ Z = U \cdot M_u $
		a následně aktualizujeme blízkost každých dvou uživatelů:
			$ U = Z \cdot M_z + U_0 $.
	\item Několikrát ziterujeme 4. krok výpočtu.
	\item Matice $Z$ obsahuje váhy popisující vhodnost zdroje pro každého uživatele.
		Matici budme procházet po řádcích (pro uživatele); setřídíme zdroje podle vah sestupně a vyfiltrujeme ty, které uživatel již sleduje (jsou v matici $M$).
		Z seznamu vezmeme prvních $k$ doporučených zdrojů a výsledný seznam uložíme pro každého uživatele do databáze.
\end{enumerate}

\subsubsection{Zálohování položky}

Kdykoli je nějaké uživatelské položce přiřazen štítek, zkontroluje se, zda štítek má nastavenou vlatnost zálohování.
Pokud ji má, vytvoří se nová zálohovací úloha, která se zařadí do fronty.
Stejně jako v případě kontroly zdrojů, i zde se fronta úloh stará o jejich postupné spouštění.
Postup zálohování jsme popsali v procesech. % TODO ref

\subsection{Veřejné rozhraní k obsahu}

Některá rozhraní aplikace jsou dostupná i neregistrovaným uživatelům

\subsubsection{Exportování položek}

Toto rozhraní poskytuje seznam položek vyhovujících seznamovému filtru, u kterého jeho vlastník povolil možnost exportování.
Export je identifikovaný kombinací aliasu a uživatelského jména vlastníka seznamového filtru.
Služba zodpovědná za výběr položek z databáze je ta samá jako v případě poskytování seznamu položek klietské položky se všemi svými důsledky.
Seznam položek je vrácen buď ve formě RSS, nebo Atom (záleží na parametru poskytnutého v URL).

\subsubsection{Zálohované položky}

Připomeňme, že během zálohování stáhneme webovou stránku, mírně ji upravíme a uložíme; záloze je přidělen klíč, který ji identifikuje.
Odkaz na zálohovanou stránku (přístupnou skrze toto rozhraní) je dostupný uživateli, který zálohu provedl ze seznamu položek.

\section{Členění klientské části}

Doposud jsme se příliš klientskou části nezabývali z důvodu, že klientská část přímo neprovádí žádné manipulace s daty; využívá jen prostředků, které jí poskytuje serverová část.
Přesto můžeme klientskou část rozčlenit do několika částí (spíše podle funkce než podle balíků):
\begin{itemize}
	\item realizace hierarchií a stromových komponent,
	\item správci, kteří komunikují se serverovou částí a ostatní správci,
	\item komponenta nekonečného seznamu,
	\item dialogy, skrze které se vytvářejí a upravují všechny objekty.
\end{itemize}

\subsection{Hierarchie}

Vzhledem ke způsobu uložení hierarchie v databázi: popisujeme cestu ke kořeni, musíme pro zobrazení stromu napřed hierarchii vybudovat.
Všichni správci, kteří poskytují hierarchii, musí každý objekt, který spravují, zpracovat a vložit do uměle vytvřené hierarchie.
Postup vypadá následovně:
\begin{enumerate}
	\item Vytvoření kořenového uzlu.
	\item Pro všechny spravované objekty:
	\item rozdělení cesty ke kořeni podle lomítek (oddělovačů jednotlivých úrovní),
	\item vytvoření uzlů pro jednotlivé úrovně, pokud ještě neexistují,
	\item vytvoření listu v nejnižší úrovni; jeho hodnotou bude příslušný objekt.
	\item Vrácení kořenového uzlu.
\end{enumerate}

Uzly se stejným rodičem jsou řazeny podle abecedy.

Z důvodu propagace změn, všechny uzly mají mohou zaregistrovat posluchače událostí změn v hierarchii: uzel byl přidán, odebrán, název uzlu se změnil.

\subsubsection{Stromové komponenty}

Klientská část aplikace obsahuje tři komponenty, které zobrazují hierarchie v podobě stromu; jsou to:
\begin{itemize}
	\item strom zdrojů -- hierarchii poskytuje správce zdrojů,
	\item strom seznamových filtrů -- hierarchii poskytuje správce seznamových filtrů,
	\item strom zpráv -- hierarchii poskytuje správce zpráv.
\end{itemize}

Zprávy ve skutečnosti netvoří strom ale prostý seznam.
Využili jsme stromovou komponentu pro jejich zobrazení z několika důvodů:
\begin{itemize}
	\item poskytuje všechny možnosti, které vyžadujeme,
	\item vzhled seznamu zpráv je blízký vzhledu stromů zdrojů a stromu seznamových filtrů,
	\item jednodušší implementace.
\end{itemize}

\subsection{Správci}

Téměř každé službě, kterou jsme popsali z pohledu serverové části odpovídá správce v klientské části.
Správce má za úkol zprostředkovávat veškerou komunikaci se serverovou částí a poskytovat nacachováná data.
Každý správce poskytuje zbytku klientské části rozhraní k metodám služby, kterou obaluje.
Tento obal většinou spočívá v zavolání příslušné metody služby a ve většině případů zobrazení hlášky o výsledku.
Hlášku o úspěchu zobrazuje v případě, že volání metody bylo způsobeno akcí uživatele.
V případě neúspěchu zobríme hlášku vždy; vždy obsahuje informaci, proč volání selhalo, například z důvodu nevyplnění některého z atributů.
Navíc zkusíme zavolat službu kontroly spojení pro zjištění, zda je dostupně spojení se serverovou částí.

V popisu jednotlivých správců budeme zmiňovat jen metody, které provádějí netriviální další akce.

\subsubsection{Správce konfigurace}

Správce konfigurace tvoří jen obálku kolem vlastní služby.
V okamžiku spuštění si stáhne veškerou konfiguraci relevantní pro přihlášeného uživatele.
Jednotlivé dotazy na konfigurační položky pak může zodpovídat okamžitě bez nutnosti komunikace se serverovou částí.

Správce konfigurace umožňuje i aktualizovat hodnoty konfiguačních položek, nicméně tuto možnost naše aplikace v současné době nevyužívá.

\subsubsection{Správce zdrojů}

Správce zdrojů načte ze serveru seznam všech uživatelských zdrojů přihlášeného uživatele.
Jednotlivé uživatelské zdroje uloží do několika slovníků, aby byly rychle vyhledatelné: podle klíče uživatelského zdroje a podle zdrojového štítku.
Dále ze všech uživatelských zdrojů vybuduje hierarchii.
Správce zdrojů si pamatuje, který zdroj je manuální (vždy existuje právě jeden).

Jelikož správce zdrojů poskytuje přístup k hierarchii uživatelských zdrojů, musí reagovat na některé události.
Jakékoli úspěšné volání metody přidat uživatelský zdroj či aktualizovat uživatelský zdroj může způsobit změnu hierarchie.
Správce zdrojů ve zmíněných metodách upravuje příslušně hierarchii uživatelských zdrojů.

\subsubsection{Správce štítků}

Správce štítků obsahuje seznam všech štítků rozdělených podle typů: uživatelský štítek, zdrojový štítek, systémový štítek.
Existují dva systémové štítky, oba reprezentují operátory: AND -- operátor \textit{a}, OR -- operátor \textit{nebo}.
Zbylé dva seznamy štítků (uživatelských a zdrojových) jsou naplněny dotazem na serverovou část při sprvním spuštění správce.
Každému štítku je nastaven zobrazovaný název: u uživatelských je to jejich název, u zdrojových je to název zdroje ke kterému patří.

Správce štítků poskytuje metodu pro vyhledávání štítků podle části jejich názvu.
Tuto metodu používá komponenta nabízení štítků v situaci, kdy uživatel může vložit název štítku (například, když přidává štítek uživatelské položce).

V případě, že uživatel přidá novou položky skrze doplněk v prohlížeči a přidá ji štítek, který doposud neexistoval, dostane se klientská část do zajímavé situace.
Seznam uživatelských položek by měl obsahovat uživatelskou položku se štítkem, o kterém nemá žádnou informaci.
Samotná uživatelská položka zná jen klíč štítku a správce štítků by jí měl objekt štítku dodat.
V takovém případě se aktualizace seznamu položek pozastaví a bude pokračovat, až si správce štítků obnoví seznam všech existujících štítků dotazem na serverovou část.

\subsubsection{Správce seznamových filtrů}

Správce seznamových filtrů poskytuje hierarchii všech seznamových filtrů, kterou si vytvoří po svém spuštění.
Dále poskytuje metody pro manipulaci se seznamovými filtry, které v případě úspěchu upraví hierarchii.
Těmito metody jsou: přidat seznamový filtr, aktualizovat seznamový filtr a odstranit seznamový filtr.

\subsubsection{Správce položek}

Správce položek poskytuje transparentní rozhraní k metodám: získat položky a přidat manuální položku.
Poslední metoda příslušné služby – aktualizovat položky – není uživateli přímo dostupná.

Správce položek si udržuje seznam změn jednotlivých uživatelských položek, které se mají provést v databázi na straně serveru.
Uživatel tento seznam změn vytváří a upravuje metodami:
\begin{itemize}
	\item štítek přidán -- přidá štítek (jeho klíč) do seznamu štítků k přidání v sadě změn příslušející uživatelské položce,
	\item štítek odebrán -- přidá štítek (jeho klíč) do seznamu štítků k odebrání v sadě změn příslušející uživatelské položce,
	\item přečtena -- nastaví stav přečtení v sadě změn příslušející uživatelské položce.
\end{itemize}
Změny se v klientské části provádějí optimisticky; předpokládá se, že provedení změny bude úspěšné.

Sada změn se chová chytře: v případě posloupnosti přidání a odebrání stejného štítku od uživatelské položky, se obě změny vyruší a příslušná sada změn se zahodí.
Periodicky, každých několik sekund probíhá dotaz obsahující seznam všech sad změn mezi správcem položek a službou serverové části.
Toto \uv{zpoždění} provedení změny v databázi nám umožňuje snížit počet dotazů po síti a zvýšit subjektivní rychlost odezvy aplikace.

\subsubsection{Správce klávesových zkratek}

Správce klávesových zkratek si při svém spuštění načte pomocí serverové služby seznam všech klávesových zkratek, která má uživatel definované.
Kombinace kláves klávesových zkratek rozparsuje a klávesové zkratky vloží do slovníku.
Dále zaregistruje posluchače událostí stisknutí klávesy: události \verb|KeyPress| a \verb|KeyDown|.
Pokud je detekováno stisknutí kláves odpovídající nějaké zkratce, vyvolá správce událost \verb|ShortcutEvent|, pomocí které informuje o klávesové zkratce své posluchače.
Sám správce klávesových zkratek zpracovává některé obecné typy zkratek: navigace mezi seznamy položek, jejich obnovu a podobné.

Správce poskytuje metody vytvořit klávesovou zkratku a odstanit klávesovou zkratku, volají službu poskytovanou serverovou částí.
Tyto metody v případě úspěchu upraví slovník klávesových zkratek.

Některé typy klávesových zkratek se odkazují na jiné objekty (štítky a seznamové filtry).
Tyto klávesové zkratky je možné vyhledat na základě klíče objektu, na který se odkazují.
Využíváme toho v dialozích s nastavením seznamových filtrů a dialogu s nastavením štítků.

\bigskip

Zbylí dva správci nejsou závislí na službách serverové části; spravují objekty, které mají význam jen v kontextu klientské části.

\subsubsection{Správce zpráv}

Zpráva je krátký text o různé důležitosti, který slouží k informování uživatele o úspěchu či selhání některé akce.

Správce zpráv je centrálním bodem v naší aplikaci pro zpracování zpráv.
Poskytuje rozhraní zbytku aplikace, skrze které může kterákoli komponenta zobrazit zprávu.
Správce zpráv také poskytuje seznam posledních zpráv v podobě hierarchie.

Správce sám zprávy nezobrazuje, umožňuje jen zaregistrovat posluchače, kteří mají být zpraveni o nových zprávách.
Jediným takovým posluchačem je vlastní grafická komponenta zodpovědná za zobrazování zpráv.


\subsubsection{Správce historie}

Framework GWT obsahuje podporu pro sledování historie v rámci aplikace.
Jednou položkou v historii je token -- řetězec reprezentující aktuální stav aplikace.
V našem případě stav aplikace odpovídá zobrazenému seznamovému filtru.

Náš správce historie rozšiřuje tento mechanismus o několik důležitých vlastností.
Pro nás je užitečnější znalost samotného seznamového filtr než jeho serializované varianty; z toho důvodu si ukládáme do seznamu stavů dvojici token -- seznamový filtr a nižší vrstvě (GWT) předáváme token.
Při reakci na událost změna historie, kterou vyvolá GWT reagujeme změnou aktuálního seznamového filtru a vyvoláním vlastní událost \verb|HistoryEvent|, skrze kterou informujeme zbytek aplikace o změně.

\subsection{Nekonečný seznam}

V této části pojmem položka myslíme jeden záznam v seznamu, který je reprezentován uživatelskou položkou.

Nejdůležitější grafickou komponentou, která tvoří naši aplikaci, je nekonečný seznam položek.
Nazýváme ho nekonečným, neboť nemá konec – jakmile uživatel odroluje seznam ke konci, načtou se další položky; takto jich může postupně zobrazit stovky.
Samozřejmé, že v případě omezeného množství položek, má seznam konec, jen může být prakticky nedosažitelný.
Výhodou tohoto přístupu je stahování ze serverové části jen takovych dat, o která má uživatel bezprostředně zájem.

Položky, které jsou v seznamu zobrazené, jsou dodávané poskytovatelem, který reaguje na událost odrolování blízko konce seznamu zavoláním metody získat položky správce položek.
Kdykoli se změní aktuální seznamový filtr, zruší se starý poskytovatel a nahradí se novým.

Nekonečný seznam je posluchačem událostí \verb|ShortcutEvent| správce zkratek; reaguje na zkratky typu: přesun na předchozí/další položku, zobrazit originální webovou stránku, přidat/odebrat štítek a podobné.

Každý řádek reprezentuje jednu uživatelskou položku se všemi jejími informacemi: zdroj ze kterého pochází, datum a čas přidání položky, senzma přiřazených štítků a hlavně titulek.
Obsah položky (text nebo HTML) se zobrazí jen v případě, že položka je vybraná a rozbalená.

\subsection{Dialogy}
Veškerá manipulace s daty v rámci aplikace probíhá v rámci dvou oblastí: v nekonečném seznamu a skrze dialogy.
V nekonečném seznamu může uživatel provádět následujíci operace:
\begin{itemize}
	\item označení uživatelské položky jako přečtené,
	\item přidání/odebrání štítku uživatelské položce, příp. vytvoření nového, pokud zadá neexistující název.
\end{itemize}

Všechny ostatní operace provádí uživatel skrze dialogy -- okna, která dočasně překryjí zbytek stránky aplikace.
Každá oblast, kterou jsme dříve zmínili, s výjimkou položek a konfigurace, nabízí jeden či více dialogů pro svou konfiguraci:
\begin{itemize}
	\item vytvoření a úpravu zdroje,
	\item vložení manuální položky,
	\item vytvoření a úpravu seznamového filtru, úpravu ad-hoc seznamového filtru,
	\item úpravu a odstranění klávesových zkratek,
	\item tvorbu klávesové zkratky typu akce,
	\item úpravu a odstranění štítku.
\end{itemize}

Každý z těchto dialogů má jinou grafickou podobu závislou na atributech spravované třídy.
V případě, že uživatel změnu provedenou v dialogu potvrdí, zavolají se metody příslušných správců, které změnu delegují na služby serverové části.

\section{Doplněk do prohlížečů}

Doplněk do prohlížeče je nepovinné rozšíření aplikace, které má cíl zjednodušit její ovládání uživatelům.
Implementace doplňku spoléhá na API poskytované platformou Crossrider, na níž je postaven.
Doplněk můžeme rozdělit do tří částí, které spolu komunikují; budeme je pro jednoduchost pojmenovávat terminologií platformy:
\begin{itemize}
	\item background -- skript, který běží v prohlížeči na pozadí; je nezávislý na otevřených stránkách; existuje vždy v jediné instanci,
	\item extension -- skript, který se vykonává zvlášť v rámci každé otevřené stránky
	\item popup -- html stránka, která se zobrazí při kliknutí na ikonku doplňku v liště prohlížeče; mezi dvěma zobrazeními se stav nezachovává.
\end{itemize}

\begin{figure}
	\centering
	\includegraphics[width=9cm]{doplnek-init.eps}
	\caption{Postup inicializace doplňku}
	\label{fig:doplnek-init}
\end{figure}
\begin{figure}
	\centering
	\includegraphics[width=11cm]{doplnek-akce.eps}
	\caption{Postup zobrazení popupu a vytvoření manuální položky}
	\label{fig:doplnek-akce}
\end{figure}

\subsection{Inicializace doplňku}

Doplněk se inicializuje při spuštění prohlížeče vykonáním background skriptu; postup je znázorněn na obrázku \ref{fig:doplnek-init}.
Nejprve zkontrolujeme, zda známe klientský token (1, 2); pokud ho známe, nastavíme akci zobrazení popup stránky při kliknutí na ikonku (3).
V případě, že token neznáme, budeme čekat na zprávu o dostupnosti tokenu (1', 2').
Až token budeme znát, nastavíme popup stránku (kroky 1, 2, 3).

Skript extension zjišťuje podle URL adresy aktuální stránku; pokud se jedná o stránku naší aplikace, podívá se do hlavičky stránky po elementu meta s názvem \verb|clientToken|.
Tento meta element do stránky přidá klientská část aplikace při své inicializaci.

\subsection{Přidání manuální položky}

Postup je znázorněn na obrázku \ref{fig:doplnek-akce}.
Uživatel spustí akci vložení aktuální stránky do naší aplikace kliknutím na ikonu doplňku v liště prohlížeče (1).
Zobrazí se vyskakovací okno s formulářem, který bude předvyplněný informacemi o aktuální stránce: titulek, URL, vybraná část textu.
Informace se získají zasláním zprávy (6) skriptu běžícímu v aktuálně otevřené stránce; skript zašle zpět (7) požadované údaje.

Formulář bude dále obsahovat předvyplněný seznam štítků a umožní přidat další štítky.
Popup zašle zprávu (2) background skriptu, který odešle dotaz (3) na serverovou část skrze API poskytované platformou Crossrider.
Background skript odpověď (4) zašle zpět (5) do popup okna.

Pokud uživatel odešle formulář, odchytí skript událost odeslání a zašle background skriptu zprávu (8) s vyplněnými údaji o manuální položce, kterou chce uživatel přidat.
Background skript odešle dotaz (9) serverové části skrze API platformy.
Využití toho API je nutné z důvodu nutnosti obejít omezení prohlížeče: skript může kontaktovat jen stránky ze stejné domény.
