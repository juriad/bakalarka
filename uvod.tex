\chaptert{Úvod}

Naše bakalářská práce se zabývá vývojem aplikace -- webové čtečky určené pro náročnější uživatele.
Kromě samozřejmé funkcionality, sledování nejrůznějších RSS kanálů, bude v sobě integrovat i několik dalších služeb, které jsou dle našeho názoru čtečkám blízké.
Myslíme si, že sjednocením vznikne celek, který bude plnit svoji úlohu lépe, než každá ze služeb odděleně.

\sectiont{Historie projektu}

Jelikož je tato práce silně inspirovaná naší zkušeností s dnes již neexistující čtečkou \href{http://www.google.com/reader/about/}{Google Reader}, popíšeme v této části alespoň stručně její historii, osud a náš vztah k ní.

\subsectiont{Používání Google Readeru}

V roce 2005 přišla společnost Google se svojí čtečkou Google Reader.
O několik let později, když jsme tuto čtečku začali používat, jsme byli s jejími možnostmi spokojení a dlouhou dobu pokrývala veškeré naše potřeby.

V roce 2011 byl nahrazen vnitřní mechanismus sdílení položek napojením na sociální síť Google+ od stejné společnosti.
Tento krok z hlediska společnosti Google dává dobrý smysl.
Propagují svoji sociální síť mezi uživateli čtečky a naopak, usnadňují používání čtečky uživatelům sociální sítě.

Z osobních důvodů ale sociální síť Google+ nechceme používat.
Důsledek změny způsobu sdílení položek se nás tedy citelně dotkl a omezil nám možnosti užití této čtečky.

Během dlouhé doby používání čtečky jsme si vypěstovali různé návyky, které nám práci s ní zefektivnily.
Bohužel jsme také narazili na její možnosti; některé operace nešlo provést dostatečně pohodlně a proto jsme je nepoužívali.
Příkladem může být funkcionalita štítků, čtečka je sice nabízela, ale k jejich přiřazení bylo nutné vypsat celý jejich název.
Pro nás to bylo daleko od ideálu: možnost nastavit štítkům klávesové zkratky a přiřazovat je stiskem jediné klávesy.

Stiskem klávesy \verb|"s"| bylo možné položce přidat či odebrat hvězdičku a klávesou \verb|"l"| bylo možné si položku oblíbit.
Nebylo možné přejít na seznam oblíbených položek klávesovou zkratkou, což tento mechanismus pro nás učinilo nepoužitelným.
Oblíbené položky byly úplně zrušeny po změně sdílení na Google+.

Způsob práce se čtečkou v posledním stádiu probíhal následovně.
Přepnuli jsme se na seznam všech položek, procházeli jej lineárně a těm položkám, které nás zaujaly, jsme přiřadili hvězdičku.
Pak jsme se přepnuli na seznam položek s hvězdičkou a procházeli je v náhodném pořadí (podle aktuální nálady).
Jednotlivé články jsme postupně četli a přečteným jsme odebrali hvězdičku.
Některé články nám v tomto seznamu vydržely dlouhé měsíce, jiné jsme vyřadili po několika málo minutách.
Nepoužívali jsme žádný mechanismus, který by umožňoval ukládat položky trvale.

\subsectiont{Konec Google Readeru}

V březnu roku 2013 společnost Google vydala prohlášení, že některé služby budou za nedlouho ukončeny.
Seznam obsahoval i oblíbenou čtečku Google Reader; její provoz bude zrušen k 1. červenci 2013~\cite{google-reader-down}.

Po tomto prohlášení začaly vznikat petice dovolávající se zachování čtečky; přestože získaly tisíce podpisů, osud čtečky nezměnily.
Zároveň začaly zaplavovat internet články obsahující seznam alternativ ke Google Readeru, služeb, ke kterým se mohou aktuální uživatelé uchýlit.

Čtečku Google Reader jsme používali do samého jejího konce.
Nechť odpočívá v pokoji.

\subsectiont{Vznik projektu}

V roce 2012, krátce po změně mechanismu sdílení, jsme zaznamenali vznik projektu \href{http://hiveminedblog.tumblr.com/}{Hivemined}.
Založil ho bývalý uživatel Google Readeru \href{mailto:apodysophilia@gmail.com}{Francis Cleary}; jeho cílem bylo vyvinout náhradu původního Google Readeru.
Postup jeho práce na této náhradě jsme sledovali, nicméně v lednu roku 2012 se odmlčel a nebylo jasné, jestli vývoj dále pokračuje či své snahy zanechal.
Na svém blogu se znovu ozval v listopadu 2012; vývoji čtečky se aktivně věnuje.
Během následujícího půlroku, těsně před uzavřením naší práce, se mu podařilo čtečku dokončit pod názvem \href{http://hivereader.com/}{Hive}.
Poslal nám e-mailem pozvánku k uzavřenému beta testování, bohužel jsme zatím neměli čas čtečku vyzkoušet a zhodnotit.

Během jeho odmlky jsme si vybrali téma bakalářské práce, u kterého jsme se zpočátku částečně inspirovali zmíněným projektem.
Zaměřili jsme se na provázání několika služeb, které dle našeho názoru umožní efektivněji používat rozšířené funkcionality čtečky.
Vycházeli jsme především ze zkušeností s několikaletým používám čtečky od Googlu.
Chtěli jsme zachovat/rozšířit její dobré vlastnosti a naopak se vyvarovat těch, které nepovažujeme za šťastně řešené.
Vzhledem k tomu, že naši aplikaci budeme sami používat, navrhli jsme ji především tak, aby nám vyhovovala.

\sectiont{Členění práce}

V tomto textu se snažíme popsat většinu aspektů aplikace, kterou jsme navrhli a vytvořili.
Aplikaci postupně rozebíráme z několika hledisek.

Nejprve v první kapitole rozčleníme čtečky podle několika kritérií, uvedeme vybrané čtečky a porovnáme je v přehledové tabulce.
Druhou kapitolu věnujeme vymezení problému, kterým se budeme dále zabývat, a jeho popisu.
Ve třetí kapitole představíme existující algoritmy doporučování a zvolíme takový, který bude v našem případě nejvhodnější.
Následující, čtvrtou kapitolu zasvětíme představení technologií a knihoven, jež při implementaci využijeme.

Vlastní návrh aplikace podrobně rozebereme v páté kapitole: její členění z pohledu architektury, s jakými entitami pracuje a jaké procesy v ní probíhají.
V šesté kapitole přiblížíme vlastní fungování aplikace, popíšeme, jaká důležitá rozhraní obsahuje a jak pobíhá komunikace jejích jednotlivých částí.
Na závěr shrneme výsledky naší práce, uvedeme možná budoucí rozšíření aplikace a svěříme se s problémy, jimž jsme museli čelit.
